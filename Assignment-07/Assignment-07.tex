% !TEX program = pdflatex
% !TEX options = -synctex=1 -interaction=nonstopmode -file-line-error "%DOC%"
% Group Theory Assignment 07
\documentclass[UTF8,10pt,a4paper]{article}
\usepackage[scheme=plain]{ctex}
\newcommand{\CourseName}{Group Theory}
\newcommand{\CourseCode}{PHYS2102}
\newcommand{\Semester}{Spring, 2020}
\newcommand{\ProjectName}{Assignment 07}
\newcommand{\DueTimeType}{Due Time}
\newcommand{\DueTime}{8:15, April 22, 2020 (Wednesday)}
\newcommand{\StudentName}{陈稼霖}
\newcommand{\StudentID}{45875852}
\usepackage[vmargin=1in,hmargin=.5in]{geometry}
\usepackage{fancyhdr}
\usepackage{lastpage}
\usepackage{calc}
\pagestyle{fancy}
\fancyhf{}
\fancyhead[L]{\CourseName}
\fancyhead[C]{\ProjectName}
\fancyhead[R]{\StudentName}
\fancyfoot[R]{\thepage\ / \pageref{LastPage}}
\setlength\headheight{12pt}
\fancypagestyle{FirstPageStyle}{
    \fancyhf{}
    \fancyhead[L]{\CourseName\\
        \CourseCode\\
        \Semester}
    \fancyhead[C]{{\Huge\bfseries\ProjectName}\\
        \DueTimeType\ : \DueTime}
    \fancyhead[R]{Name : \makebox[\widthof{\StudentID}][s]{\StudentName}\\
        Student ID\@ : \StudentID\\
        Score : \underline{\makebox[\widthof{\StudentID}]{}}}
    \fancyfoot[R]{\thepage\ / \pageref{LastPage}}
    \setlength\headheight{36pt}
}
\usepackage{amsmath,amssymb,amsthm,bm}
\allowdisplaybreaks[4]
\newtheoremstyle{Problem}
{}
{}
{}
{}
{\bfseries}
{.}
{ }
{\thmname{#1}\thmnumber{ #2}\thmnote{ (#3)} Score: \underline{\qquad\qquad}}
\theoremstyle{Problem}
\newtheorem{prob}{Problem}
\newtheoremstyle{Solution}
{}
{}
{}
{}
{\bfseries}
{:}
{ }
{\thmname{#1}}
\makeatletter
\def\@endtheorem{\qed\endtrivlist\@endpefalse}
\makeatother
\theoremstyle{Solution}
\newtheorem*{sol}{Solution}
% \usepackage{graphicx}
\begin{document}
\thispagestyle{FirstPageStyle}
Consider a particle of mass $\mu$ confined to a square in two dimensions whose vertices are located at $(z,x)=(1,1)$, $(1,-1)$, $(-1,-1)$, and $(-1,1)$ on the $zOx$ plane. The potential is zero within the square and infinite on the edge of the square. The eigenfunctions $\psi_{mn}(z,x)$ of the Hamiltonian of the particle are of the form
\[
    \psi_{mn}(z,x)\propto\left\{\begin{array}{ll}
        \cos(k_mz)\cos(k_nx),&\text{if both $m$ and $n$ are odd},\\
        \cos(k_mz)\sin(k_nx),&\text{if $m$ is odd but $n$ is even},\\
        \sin(k_mz)\cos(k_nx),&\text{if $m$ is even but $n$ is odd},\\
        \sin(k_mz)\sin(k_nx),&\text{if both $m$ and $n$ are even},
    \end{array}\right.
\]
where $k_m=m\pi/2$, $k_n=n\pi/2$, and $m$ and $n$ are positive integers. The corresponding eigenvalues are given by
\[
    E_{mn}=\frac{\pi^2\hbar^2}{8\mu}(m^2+n^2).
\]
The symmetry group of the Hamiltonian $H_0$ is $D_4$ whose character table is given by
\begin{table}[h]
    \centering
    \begin{tabular}{c|ccccc}
     & $C_1=\{E\}$ & $C_2=\{C_{2x},C_{2z}\}$ & $C_3=\{C_{2y}\}$ & $C_4=\{C_{4y},C_{4y}^{-1}\}$ & $C_5=\{C_{2c},C_{2d}\}$ \\ \hline
    $\Gamma^1$ & $1$ & $1$ & $1$ & $1$ & $1$ \\
    $\Gamma^2$ & $1$ & $1$ & $1$ & $-1$ & $-1$ \\
    $\Gamma^3$ & $1$ & $-1$ & $1$ & $1$ & $-1$ \\
    $\Gamma^4$ & $1$ & $-1$ & $1$ & $-1$ & $1$ \\
    $\Gamma^5$ & $2$ & $0$ & $-2$ & $0$ & $0$
    \end{tabular}
\end{table}
\begin{prob}
    For which irreducible representations do the eigenfunctions $\psi_{11}(z,x)$ and $\psi_{22}(z,x)$ form bases respectively?
\end{prob}
\begin{sol}
    Suppose the dimension of irreducible representation $\Gamma^p$ is $d_p$. Since the order of $D_4$ is $8$, we have
    \begin{gather}
        d_1^2+d_2^2+d_3^2+d_4^2+d_5^2=8,\\
        \Longrightarrow d_1=d_2=d_3=d_4=1,\quad d_5=2.
    \end{gather}
    $\Gamma_1$, $\Gamma_2$, $\Gamma_3$, and $\Gamma_4$ are $1$-dimensional representations and $\Gamma_5$ is $2$-dimensional representation.\\
    For $1$-dimensional representations $\Gamma_1$, $\Gamma_2$, $\Gamma_3$, and $\Gamma_4$ of $D_4$, the basis functions transform for every coordinate transformation $T$ of $D_4$ according to
    \begin{align}
        Q(T)\psi(\vec{r})=\Gamma(T)_{11}\psi(\vec{r})=\chi(T)\psi.
    \end{align}
    We first calculate $Q(T)\psi_{11}(z,x)$ and $Q(T)\psi_{22}(z,x)$ for every coordinate transformation $T$ of $D_4$. We already know the transformation matrices of $D_4$ in two dimensions are
    \begin{align*}
        R(E)=&\left(\begin{matrix}
            1&0\\
            0&1
        \end{matrix}\right),&R(C_{2x})=&\left(\begin{matrix}
            1&0\\
            0&-1
        \end{matrix}\right),&R(C_{2y})=&\left(\begin{matrix}
            -1&0\\
            0&-1
        \end{matrix}\right),&R(C_{2z})=&\left(\begin{matrix}
            -1&0\\
            0&1
        \end{matrix}\right),\\
        R(C_{4y})=&\left(\begin{matrix}
            0&-1\\
            1&0
        \end{matrix}\right),&R(C_{4y}^{-1})=&\left(\begin{matrix}
            0&1\\
            -1&0
        \end{matrix}\right),&R(C_{2c})=&\left(\begin{matrix}
            0&1\\
            1&0
        \end{matrix}\right),&R(C_{2d})=&\left(\begin{matrix}
            0&-1\\
            -1&0
        \end{matrix}\right).
    \end{align*}
    Since
    \begin{align}
        R(T)^{-1}=R(T)^T
    \end{align}
    for every $T\in G$, we have
    \begin{align}
        Q(T)\psi(\vec{r})=\psi(R(T)^{-1}\vec{r})=\psi(R(T)^T\vec{r}).
    \end{align}
    Since
    \begin{align}
        R(E)^T\vec{r}=&\left(\begin{matrix}
            1&0\\
            0&1
        \end{matrix}\right)\left(\begin{matrix}
            z\\
            x
        \end{matrix}\right)=\left(\begin{matrix}
            z\\
            x
        \end{matrix}\right),\\
        R(C_{2x})^T\vec{r}=&\left(\begin{matrix}
            1&0\\
            0&-1
        \end{matrix}\right)\left(\begin{matrix}
            z\\
            x
        \end{matrix}\right)=\left(\begin{matrix}
            z\\
            -x
        \end{matrix}\right),\\
        R(C_{2y})^T\vec{r}=&\left(\begin{matrix}
            -1&0\\
            0&-1
        \end{matrix}\right)\left(\begin{matrix}
            z\\
            x
        \end{matrix}\right)=\left(\begin{matrix}
            -z\\
            -x
        \end{matrix}\right),\\
        R(C_{2z})^T\vec{r}=&\left(\begin{matrix}
            -1&0\\
            0&1
        \end{matrix}\right)\left(\begin{matrix}
            z\\
            x
        \end{matrix}\right)=\left(\begin{matrix}
            -z\\
            x
        \end{matrix}\right),\\
        R(C_{4y})^T\vec{r}=&\left(\begin{matrix}
            0&1\\
            -1&0
        \end{matrix}\right)\left(\begin{matrix}
            z\\
            x
        \end{matrix}\right)=\left(\begin{matrix}
            x\\
            -z
        \end{matrix}\right),\\
        R(C_{4y}^{-1})^T\vec{r}=&\left(\begin{matrix}
            0&-1\\
            1&0
        \end{matrix}\right)\left(\begin{matrix}
            z\\
            x
        \end{matrix}\right)=\left(\begin{matrix}
            -x\\
            z
        \end{matrix}\right),\\
        R(C_{2c})^T\vec{r}=&\left(\begin{matrix}
            0&1\\
            1&0
        \end{matrix}\right)\left(\begin{matrix}
            z\\
            x
        \end{matrix}\right)=\left(\begin{matrix}
            x\\
            z
        \end{matrix}\right),\\
        R(C_{2d})^T\vec{r}=&\left(\begin{matrix}
            0&-1\\
            -1&0
        \end{matrix}\right)\left(\begin{matrix}
            z\\
            x
        \end{matrix}\right)=\left(\begin{matrix}
            -x\\
            -z
        \end{matrix}\right),
    \end{align}
    for $\psi_{11}(z,x)$, we have
    \begin{align}
        \nonumber&Q(E)\psi_{11}(z,x)=R(C_{2x})\psi_{11}(z,x)=R(C_{2y})\psi_{11}(z,x)=R(C_{2z})\psi_{11}(z,x)\\
        =&Q(C_{4y})\psi_{11}(z,x)=Q(C_{4y}^{-1})\psi_{11}(z,x)=Q(C_{2c})\psi_{11}(z,x)=Q(C_{2d})\psi_{11}(z,x)=\psi_{11}(z,x)=\sin(k_1z)\sin(k_2x)=\psi_{11}(z,x),
    \end{align}
    \begin{align}
        \Longrightarrow\chi(C_1)=\chi(C_2)=\chi(C_3)=\chi(C_4)=\chi(C_5)=1.
    \end{align}
    and for $\psi_{22}(z,x)$, we have
    \begin{align}
        Q(E)\psi_{22}(z,x)=Q(C_{2y})\psi_{22}(z,x)=Q(C_{2c})\psi_{22}(z,x)=Q(C_{2d})\psi_{22}(z,x)=&\sin(k_1z)\sin(k_1x)=\psi_{22}(z,x),\\
        Q(C_{2x})\psi_{22}(z,x)=Q(C_{2z})\psi_{22}(z,x)=Q(C_{4y})\psi_{22}(z,x)=Q(C_{4y}^{-1})\psi_{22}(z,x)=&-\sin(k_1z)\sin(k_1x)=-\psi_{22}(z,x).
    \end{align}
    \begin{align}
        \Longrightarrow\chi(C_1)=\chi(C_3)=\chi(C_5)=1,\quad\chi(C_2)=\chi(C_4)=-1.
    \end{align}
    Therefore, the eigenfunction $\psi_{11}(z,x)$ forms the basis of $\Gamma_1$ and the eigenfunction $\psi_{22}(z,x)$ forms the basis of $\Gamma_4$.
\end{sol}

\begin{prob}
    Find the matrices representing all the elements of $D_4$ in the space spanned by the degenerate eigenfunctions $\psi_{12}(z,x)$ and $\psi_{21}(z,x)$. And then calculate the characters for all the classes of $D_4$ in this representation. For which irreducible representation do $\psi_{12}(z,x)$ and $\psi_{21}(z,x)$ form a basis?
\end{prob}
\begin{sol}
    Since
    \begin{align}
        Q(E)\psi_{12}(z,x)=&\cos(k_1z)\sin(k_2x)=\psi_{12}(z,x)=\Gamma(E)_{11}\psi_{12}(z,x)+\Gamma(E)_{21}\psi_{21}(z,x),\\
        Q(E)\psi_{21}(z,x)=&\sin(k_2z)\cos(k_1x)=\psi_{21}(z,x)=\Gamma(E)_{12}\psi_{12}(z,x)+\Gamma(E)_{22}\psi_{21}(z,x),
    \end{align}
    we have
    \begin{gather}
        \Gamma(E)_{11}=1,\quad\Gamma(E)_{21}=0,\quad\Gamma(E)_{12}=0,\quad\Gamma(E)_{22}=1,\\
        \Longrightarrow\Gamma(E)=\left(\begin{matrix}
            1&0\\
            0&1
        \end{matrix}\right).
    \end{gather}
    Since
    \begin{align}
        Q(C_{2x})\psi_{12}(z,x)=&-\cos(k_1z)\sin(k_2x)=-\psi_{12}(z,x)=\Gamma(C_{2x})_{11}\psi_{12}(z,x)+\Gamma(C_{2x})_{21}\psi_{21}(z,x),\\
        Q(C_{2x})\psi_{21}(z,x)=&\sin(k_2z)\cos(k_1x)=\psi_{21}(z,x)=\Gamma(C_{2x})_{12}\psi_{12}(z,x)+\Gamma(C_{2x})_{22}\psi_{21}(z,x),
    \end{align}
    we have
    \begin{gather}
        \Gamma(C_{2x})_{11}=-1,\quad\Gamma(C_{2x})_{21}=0,\quad\Gamma(C_{2x})_{12}=0,\quad\Gamma(C_{2x})_{22}=1,\\
        \Longrightarrow\Gamma(C_{2x})=\left(\begin{matrix}
            -1&0\\
            0&1
        \end{matrix}\right).
    \end{gather}
    Since
    \begin{align}
        Q(C_{2y})\psi_{12}(z,x)=&-\cos(k_1z)\sin(k_2x)=-\psi_{12}(z,x)=\Gamma(C_{2y})_{11}\psi_{12}(z,x)+\Gamma(C_{2y})_{21}\psi_{21}(z,x),\\
        Q(C_{2y})\psi_{21}(z,x)=&-\sin(k_2z)\cos(k_1x)=-\psi_{21}(z,x)=\Gamma(C_{2z})_{12}\psi_{12}(z,x)+\Gamma(C_{2y})_{22}\psi_{21}(z,x),
    \end{align}
    we have
    \begin{gather}
        \Gamma(C_{2y})_{11}=-1,\quad\Gamma(C_{2y})_{21}=0,\quad\Gamma(C_{2y})_{12}=0,\quad\Gamma(C_{2y})_{22}=-1,\\
        \Longrightarrow\Gamma(C_{2y})=\left(\begin{matrix}
            -1&0\\
            0&-1
        \end{matrix}\right).
    \end{gather}
    Since
    \begin{align}
        Q(C_{2z})\psi_{12}(z,x)=&\cos(k_1z)\sin(k_2x)=\psi_{12}(z,x)=\Gamma(C_{2z})_{11}\psi_{12}(z,x)+\Gamma(C_{2z})_{21}\psi_{21}(z,x),\\
        Q(C_{2z})\psi_{21}(z,x)=&-\sin(k_2z)\cos(k_1x)=-\psi_{21}(z,x)=\Gamma(C_{2z})_{12}\psi_{12}(z,x)+\Gamma(C_{2z})_{22}\psi_{21}(z,x),
    \end{align}
    we have
    \begin{gather}
        \Gamma(C_{2z})_{11}=1,\quad\Gamma(C_{2z})_{21}=0,\quad\Gamma(C_{2z})_{12}=0,\quad\Gamma(C_{2z})_{22}=-1,\\
        \Longrightarrow\Gamma(C_{2z})=\left(\begin{matrix}
            1&0\\
            0&-1
        \end{matrix}\right).
    \end{gather}
    Since
    \begin{align}
        Q(C_{4y})\psi_{12}(z,x)=&-\cos(k_1x)\sin(k_2z)=-\psi_{21}(z,x)=\Gamma(C_{4y})_{11}\psi_{12}(z,x)+\Gamma(C_{4y})_{21}\psi_{21}(z,x),\\
        Q(C_{4y})\psi_{21}(z,x)=&\sin(k_2x)\cos(k_1z)=\psi_{12}(z,x)=\Gamma(C_{4y})_{12}\psi_{12}(z,x)+\Gamma(C_{4y})_{22}\psi_{21}(z,x),
    \end{align}
    we have
    \begin{gather}
        \Gamma(C_{4y})_{11}=0,\quad\Gamma(C_{4y})_{21}=-1,\quad\Gamma(C_{4y})_{12}=1,\quad\Gamma(C_{4y})_{22}=0,\\
        \Longrightarrow\Gamma(C_{4y})=\left(\begin{matrix}
            0&1\\
            -1&0
        \end{matrix}\right).
    \end{gather}
    Since
    \begin{align}
        Q(C_{4y}^{-1})\psi_{12}(z,x)=&\cos(k_1x)\sin(k_2z)=\psi_{21}(z,x)=\Gamma(C_{4y}^{-1})_{11}\psi_{12}(z,x)+\Gamma(C_{4y}^{-1})_{21}\psi_{21}(z,x),\\
        Q(C_{4y}^{-1})\psi_{21}(z,x)=&-\sin(k_2x)\cos(k_1z)=-\psi_{12}(z,x)=\Gamma(C_{4y}^{-1})_{12}\psi_{12}(z,x)+\Gamma(C_{4y}^{-1})_{22}\psi_{21}(z,x),
    \end{align}
    we have
    \begin{gather}
        \Gamma(C_{4y}^{-1})_{11}=0,\quad\Gamma(C_{4y}^{-1})_{21}=1,\quad\Gamma(C_{4y}^{-1})_{12}=-1,\quad\Gamma(C_{4y}^{-1})_{22}=0,\\
        \Longrightarrow\Gamma(C_{4y}^{-1})=\left(\begin{matrix}
            0&-1\\
            1&0
        \end{matrix}\right).
    \end{gather}
    Since
    \begin{align}
        Q(C_{2c})\psi_{12}(z,x)=&\cos(k_1x)\sin(k_2z)=\psi_{21}(z,x)=\Gamma(C_{2c})_{11}\psi_{12}(z,x)+\Gamma(C_{2c})_{21}\psi_{21}(z,x),\\
        Q(C_{2c})\psi_{21}(z,x)=&\sin(k_2x)\cos(k_1z)=\psi_{12}(z,x)=\Gamma(C_{2c})_{12}\psi_{12}(z,x)+\Gamma(C_{2c})_{22}\psi_{21}(z,x),
    \end{align}
    we have
    \begin{gather}
        \Gamma(C_{2c})_{11}=0,\quad\Gamma(C_{2c})_{21}=1,\quad\Gamma(C_{2c})_{12}=1,\Gamma(C_{2c})_{22}=0,\\
        \Longrightarrow\Gamma(C_{2c})=\left(\begin{matrix}
            0&1\\
            1&0
        \end{matrix}\right).
    \end{gather}
    Since
    \begin{align}
        Q(C_{2d})\psi_{12}(z,x)=&-\cos(k_1x)\sin(k_2z)=-\psi_{21}(z,x)=\Gamma(C_{2d})_{11}\psi_{12}(z,x)+\Gamma(C_{2d})_{21}\psi_{21}(z,x),\\
        Q(C_{2d})\psi_{21}(z,x)=&-\sin(k_2x)\cos(k_1z)=-\psi_{12}(z,x)=\Gamma(C_{2d})_{12}\psi_{12}(z,x)+\Gamma(C_{2d})_{22}\psi_{21}(z,x),
    \end{align}
    we have
    \begin{gather}
        \Gamma(C_{2d})_{11}=0,\quad\Gamma(C_{2d})_{21}=-1,\quad\Gamma(C_{2d})_{12}=-1,\quad\Gamma(C_{2d})_{22}=0,\\
        \Longrightarrow\Gamma(C_{2d})=\left(\begin{matrix}
            0&-1\\
            -1&0
        \end{matrix}\right).
    \end{gather}
    The characters for all the classes of $D_4$ in this representation are
    \begin{align}
        \chi(C_1)=2,\quad\chi(C_2)=\chi(C_4)=\chi(C_5)=0,\quad\chi(C_3)=-2.
    \end{align}
    Therefore, $\psi_{12}(z,x)$ and $\psi_{21}(z,x)$ form a basis of $\Gamma_5$.
\end{sol}

\begin{prob}
    What is the degeneracy corresponding to $(m=6,n=7)$ and $(m=2,n=9)$? Is this degeneracy normal or accidental?
\end{prob}
\begin{sol}
    Similar to last problem, both $(m=6,n=7)$ and $(m=2,n=9)$ form the basis of the representation $\Gamma^5$ respectively, so their corresponding representation is irreducible, $\Gamma=\Gamma^5\oplus\Gamma^5$. Therefore, the corresponding degeneracy is accidental.
\end{sol}

\begin{prob}
    Find the matrices representing all the elements of $D_4$ in the space spanned by the degenerate eigenfunctions $\psi_{mn}(z,x)$ and $\psi_{nm}(z,x)$. Here both $m$ and $n$ are odd integers but they are not equal. And then calculate the characters for all the classes of $D_4$ in this representation. Is this representation reducible or irreducible? If this representation is reducible, write it as a direct sum of irreducible representations.
\end{prob}
\begin{sol}
    Since
    \begin{align}
        Q(E)\psi_{mn}(z,x)=&\cos(k_mz)\cos(k_nx)=\psi_{mn}(z,x)=\Gamma(E)_{11}\psi_{mn}(z,x)+\Gamma(E)_{21}\psi_{nm}(z,x),\\
        Q(E)\psi_{nm}(z,x)=&\cos(k_nz)\cos(k_mx)=\psi_{nm}(z,x)=\Gamma(E)_{12}\psi_{mn}(z,x)+\Gamma(E)_{22}\psi_{nm}(z,x),
    \end{align}
    we have
    \begin{gather}
        \Gamma(E)_{11}=1,\quad\Gamma(E)_{21}=0,\quad\Gamma(E)_{12}=0,\quad\Gamma(E)_{22}=1,\\
        \Longrightarrow\Gamma(E)=\left(\begin{matrix}
            1&0\\
            0&1
        \end{matrix}\right).
    \end{gather}
    Since
    \begin{align}
        Q(C_{2x})\psi_{mn}(z,x)=&\cos(k_mz)\cos(k_nx)=\psi_{mn}(z,x)=\Gamma(C_{2x})_{11}\psi_{mn}(z,x)+\Gamma(C_{2x})_{21}\psi_{nm}(z,x),\\
        Q(C_{2y})\psi_{nm}(z,x)=&\cos(k_nz)\cos(k_mx)=\psi_{nm}(z,x)=\Gamma(C_{2x})_{12}\psi_{mn}(z,x)+\Gamma(C_{2x})_{22}\psi_{nm}(z,x),
    \end{align}
    we have
    \begin{gather}
        \Gamma(C_{2x})_{11}=1,\quad\Gamma(C_{2x})_{21}=0,\quad\Gamma(C_{2x})_{12}=0,\quad\Gamma(C_{2x})_{22}=1,\\
        \Longrightarrow\Gamma(C_{2x})=\left(\begin{matrix}
            1&0\\
            0&1
        \end{matrix}\right).
    \end{gather}
    Since
    \begin{align}
        Q(C_{2y})\psi_{mn}(z,x)=&\cos(k_mz)\cos(k_nx)=\psi_{mn}(z,x)=\Gamma(C_{2y})_{11}\psi_{mn}(z,x)+\Gamma(C_{2y})_{21}\psi_{nm}(z,x),\\
        Q(C_{2y})\psi_{nm}(z,x)=&\cos(k_nz)\cos(k_mx)=\psi_{nm}(z,x)=\Gamma(C_{2y})_{12}\psi_{mn}(z,x)+\Gamma(C_{2y})_{22}\psi_{nm}(z,x),
    \end{align}
    we have
    \begin{gather}
        \Gamma(C_{2y})_{11}=1,\quad\Gamma(C_{2y})_{21}=0,\quad\Gamma(C_{2y})_{12}=0,\quad\Gamma_{22}(C_{2y})=1,\\
        \Longrightarrow\Gamma(C_{2y})=\left(\begin{matrix}
            1&0\\
            0&1
        \end{matrix}\right).
    \end{gather}
    Since
    \begin{align}
        Q(C_{2z})\psi_{mn}(z,x)=&\cos(k_mz)\cos(k_nx)=\psi_{mn}(z,x)=\Gamma(C_{2z})_{11}\psi_{mn}(z,x)+\Gamma(C_{2z})_{21}\psi_{nm}(z,x),\\
        Q(C_{2z})\psi_{nm}(z,x)=&\cos(k_nz)\cos(k_mx)=\psi_{nm}(z,x)=\Gamma(C_{2z})_{12}\psi_{mn}(z,x)+\Gamma(C_{2z})_{12}\psi_{nm}(z,x),
    \end{align}
    we have
    \begin{gather}
        \Gamma(C_{2z})_{11}=1,\quad\Gamma(C_{2z})_{21}=0,\quad\Gamma(C_{2z})_{12}=0,\quad\Gamma(C_{2z})_{22}=1,\\
        \Longrightarrow\Gamma(C_{2z})=\left(\begin{matrix}
            1&0\\
            0&1
        \end{matrix}\right).
    \end{gather}
    Since
    \begin{align}
        Q(C_{4y})\psi_{mn}(z,x)=&\cos(k_mx)\cos(k_nz)=\psi_{nm}(z,x)=\Gamma(C_{4y})_{11}\psi_{mn}(z,x)+\Gamma(C_{4y})_{21}\psi_{nm}(z,x),\\
        Q(C_{4y})\psi_{nm}(z,x)=&\cos(k_nx)\cos(k_mz)=\psi_{mn}(z,x)=\Gamma(C_{4y})_{12}\psi_{mn}(z,x)+\Gamma(C_{4y})_{22}\psi_{nm}(z,x),
    \end{align}
    we have
    \begin{gather}
        \Gamma(C_{4y})_{11}=0,\quad\Gamma(C_{4y})_{21}=1,\quad\Gamma(C_{4y})_{12}=1,\quad\Gamma(C_{4y})_{22}=0,\\
        \Longrightarrow\Gamma(C_{4y})=\left(\begin{matrix}
            0&1\\
            1&0
        \end{matrix}\right).
    \end{gather}
    Since
    \begin{align}
        Q(C_{4y}^{-1})\psi_{mn}(z,x)=&\cos(k_mx)\cos(k_nz)=\psi_{nm}(z,x)=\Gamma(C_{4y}^{-1})_{11}\psi_{mn}(z,x)+\Gamma(C_{4y}^{-1})_{21}\psi_{nm}(z,x),\\
        Q(C_{4y}^{-1})\psi_{nm}(z,x)=&\cos(k_nx)\cos(k_mz)=\psi_{mn}(z,x)=\Gamma(C_{4y}^{-1})_{12}\psi_{mn}(z,x)+\Gamma(C_{4y}^{-1})_{22}\psi_{nm}(z,x),
    \end{align}
    we have
    \begin{gather}
        \Gamma(C_{4y}^{-1})_{11}=-1,\quad\Gamma(C_{4y}^{-1})_{21}=1,\quad\Gamma(C_{4y}^{-1})_{12}=1,\quad\Gamma(C_{4y}^{-1})_{22}=0,\\
        \Longrightarrow\Gamma(C_{4y}^{-1})=\left(\begin{matrix}
            0&1\\
            1&0
        \end{matrix}\right).
    \end{gather}
    Since
    \begin{align}
        Q(C_{2c})\psi_{mn}(z,x)=&\cos(k_mx)\cos(k_nz)=\psi_{nm}(z,x)=\Gamma(C_{2c})_{11}\psi_{mn}(z,x)+\Gamma(C_{2c})_{21}\psi_{nm}(z,x),\\
        Q(C_{2c})\psi_{nm}(z,x)=&\cos(k_nx)\cos(k_mz)=\psi_{mn}(z,x)=\Gamma(C_{2c})_{12}\psi_{mn}(z,x)+\Gamma(C_{2c})_{22}\psi_{nm}(z,x),
    \end{align}
    we have
    \begin{gather}
        \Gamma(C_{2c})_{11}=0,\quad\Gamma(C_{2c})_{21}=1,\quad\Gamma(C_{2c})_{12}=1,\quad\Gamma(C_{2c})_{22}=0,\\
        \Longrightarrow\Gamma(C_{2c})=\left(\begin{matrix}
            0&1\\
            1&0
        \end{matrix}\right).
    \end{gather}
    Since
    \begin{align}
        Q(C_{2c})\psi_{mn}(z,x)=&\cos(k_mx)\cos(k_nz)=\psi_{nm}(z,x)=\Gamma(C_{2d})_{11}\psi_{mn}(z,x)+\Gamma(C_{2d})_{21}\psi_{nm}(z,x),\\
        Q(C_{2c})\psi_{nm}(z,x)=&\cos(k_nx)\cos(k_mz)=\psi_{mn}(z,x)=\Gamma(C_{2d})_{12}\psi_{mn}(z,x)+\Gamma(C_{2d})_{22}\psi_{nm}(z,x),
    \end{align}
    we have
    \begin{gather}
        \Gamma(C_{2d})_{11}=0,\quad\Gamma(C_{2d})_{21}=1,\quad\Gamma(C_{2d})_{12}=1,\quad\Gamma(C_{2d})_{22}=0,\\
        \Longrightarrow\Gamma(C_{2d})=\left(\begin{matrix}
            0&1\\
            1&0
        \end{matrix}\right).
    \end{gather}
    The characters for all the classes of $D_4$ in this representation is
    \begin{align}
        \chi(C_1)=\chi(C_2)=\chi(C_3)=2,\quad\chi(C_4)=\chi(C_5)=0.
    \end{align}
    This representation is similar to such a representation
    \begin{align}
        \Gamma'=S^{-1}\Gamma S=\frac{\sqrt{2}}{2}\left(\begin{matrix}
            1&1\\
            1&-1
        \end{matrix}\right)\Gamma\frac{\sqrt{2}}{2}\left(\begin{matrix}
            1&1\\
            1&-1
        \end{matrix}\right)
    \end{align}
    that
    \begin{align}
        \Gamma'(E)=\Gamma'(C_{2x})=\Gamma'(C_{2y})=\Gamma'(C_{2z})=&\left(\begin{matrix}
            1&0\\
            0&1
        \end{matrix}\right),\\
        \Gamma'(C_{4y})=\Gamma'(C_{4y}^{-1})=\Gamma'(C_{2c})=\Gamma'(C_{2d})=&\left(\begin{matrix}
            1&0\\
            0&-1
        \end{matrix}\right).
    \end{align}
    Therefore, this representation is reducible:
    \begin{align}
        \Gamma\cong\Gamma_1\oplus\Gamma_2.
    \end{align}
\end{sol}

\begin{prob}
    Consider the case in which the particle is subject to an interaction given by $Ax$ with $A$ a constant.
    \begin{enumerate}
        \item[(a)] For which irreducible representation of $D_4$ is $x$ an irreducible tensor operator?
        \item[(b)] Consider the transitions caused by the interaction. If the particle is initially in the state $\psi_{mn}(z,x)$ or $\psi_{nm}(z,x)$ with $m$ and $n$ respectively even and odd integers, through reducing the direct product of irreducible representations find the irreducible representations which the allowed final state transform as.
    \end{enumerate}
\end{prob}
\begin{sol}
    \begin{enumerate}
        \item[(a)] Let $Q(T)xQ(T)^{-1}$ operate on an arbitrary wavefunction $f(z,x)$, we have
        \begin{align}
            Q(T)xQ(T^{-1})f(z,x)=Q(T)\{x[Q(T)^{-1}\psi(z,x)]\}=[Q(T)x][Q(X)Q(X)^{-1}\psi(z,x)]=[Q(T)x]\psi(z,x).
        \end{align}
        Due to the arbitrariness of the wavefunction $\psi(z,x)$, we have
        \begin{align}
            Q(T)xQ(T)^{-1}=Q(T)x.
        \end{align}
        Now that $x$ is an irreducible tensor operator, let $z$ also be in the set of irreducible operators. To make $x$ an irreducible tensor operator, we need
        \begin{align}
            Q(T)x=&\Gamma^q(T)_{12}z+\Gamma^q(T)_{22}x,\\
            Q(T)z=&\Gamma^q(T)_{11}z+\Gamma^q(T)_{21}x.
        \end{align}
        for every $T\in D_4$.\\
        For $T=E$, we need
        \begin{align}
            Q(E)x=&x=\Gamma^q(E)_{12}z+\Gamma^q(E)_{22}x,\\
            Q(E)z=&z=\Gamma^q(E)_{11}z+\Gamma^q(E)_{21}x,
        \end{align}
        \begin{gather}
            \Longrightarrow\Gamma^q(E)_{12}=0,\quad\Gamma^q(E)_{22}=1,\quad\Gamma^q(E)_{11}=1,\quad\Gamma^q(E)_{11}=0,\\
            \Longrightarrow\Gamma^q(E)=\left(\begin{matrix}
                1&0\\
                0&1
            \end{matrix}\right).
        \end{gather}
        For $T=C_{2x}$, we need
        \begin{align}
            Q(C_{2x})x=&-x=\Gamma^q(C_{2x})_{12}z+\Gamma^q(C_{2x})_{22}x,\\
            Q(C_{2x})z=&z=\Gamma^q(C_{2x})_{11}z+\Gamma^q(C_{2x})_{21}x,
        \end{align}
        \begin{gather}
            \Longrightarrow\Gamma^q(C_{2x})_{12}=0,\quad\Gamma^q(C_{2z})_{22}=-1,\quad\Gamma^q(C_{2x})_{11}=1,\quad\Gamma^q(C_{2x})_{21}=0\\
            \Longrightarrow\Gamma^q(C_{2x})=\left(\begin{matrix}
                1&0\\
                0&-1
            \end{matrix}\right).
        \end{gather}
        For $T=C_{2y}$, we need
        \begin{align}
            Q(C_{2y})x=&-x=\Gamma^q(C_{2y})_{12}z+\Gamma^q(C_{2y})_{22}x,\\
            Q(C_{2y})z=&-z=\Gamma^q(C_{2y})_{11}z+\Gamma^q(C_{2y})_{21}x,
        \end{align}
        \begin{gather}
            \Longrightarrow\Gamma^q(C_{2y})_{12}=0,\quad\Gamma^q(C_{2y})_{22}=-1,\quad\Gamma^q(C_{2y})_{11}=-1,\quad\Gamma^q(C_{2y})_{21}=0,\\
            \Longrightarrow\Gamma^q(C_{2y})=\left(\begin{matrix}
                -1&0\\
                0&-1
            \end{matrix}\right).
        \end{gather}
        For $T=C_{2z}$, we need
        \begin{align}
            Q(C_{2z})x=&x=\Gamma^q(C_{2z})_{12}z+\Gamma^q(C_{2z})_{22}x,\\
            Q(C_{2z})z=&-z=\Gamma^q(C_{2z})_{11}z+\Gamma^q(C_{2z})_{21}x,
        \end{align}
        \begin{gather}
            \Longrightarrow\Gamma^q(C_{2z})_{12}=0,\quad\Gamma^q(C_{2z})_{22}=1,\quad\Gamma^q(C_{2z})_{11}=-1,\quad\Gamma^q(C_{2z})_{12}=0,\\
            \Longrightarrow\Gamma^q(C_{2z})=\left(\begin{matrix}
                -1&0\\
                0&1
            \end{matrix}\right).
        \end{gather}
        For $T=C_{4y}$, we need
        \begin{align}
            Q(C_{4y})x=&-z=\Gamma^q(C_{4y})_{12}z+\Gamma^q(C_{4y})_{22}x,\\
            Q(C_{4y})z=&x=\Gamma^q(C_{4y})_{11}z+\Gamma^q(C_{4y})_{21}x,
        \end{align}
        \begin{gather}
            \Longrightarrow\Gamma^q(C_{4y})_{12}=-1,\quad\Gamma^q(C_{4y})_{22}=0,\quad\Gamma^q(C_{4y})_{11}=0,\quad\Gamma^q(C_{4y})_{21}=1,\\
            \Longrightarrow\Gamma^q(C_{4y})=\left(\begin{matrix}
                0&-1\\
                1&0
            \end{matrix}\right).
        \end{gather}
        For $T=C_{4y}^{-1}$, we need
        \begin{align}
            Q(C_{4y}^{-1})=&z=\Gamma^q(C_{4y}^{-1})_{12}z+\Gamma^q(C_{4y}^{-1})_{22}x,\\
            Q(C_{4y}^{-1})=&-x=\Gamma^q(C_{4y}^{-1})_{11}z+\Gamma^q(C_{4y}^{-1})_{21}x,
        \end{align}
        \begin{gather}
            \Longrightarrow\Gamma^q(C_{4y})_{12}=1,\quad\Gamma^q(C_{4y})_{22}=0,\quad\Gamma^q(C_{4y}^{-1})_{11}=0,\quad\Gamma^q(C_{4y}^{-1})_{21}=-1,\\
            \Longrightarrow\Gamma^q(C_{4y}^{-1})=\left(\begin{matrix}
                0&1\\
                -1&0
            \end{matrix}\right).
        \end{gather}
        For $T=C_{2c}$, we need
        \begin{align}
            Q(C_{2c})x=z=\Gamma^q(C_{2c})_{12}z+\Gamma^q(C_{2c})_{22}x,\\
            Q(C_{2c})x=x=\Gamma^q(C_{2c})_{11}z+\Gamma^q(C_{2c})_{21}x,
        \end{align}
        \begin{gather}
            \Longrightarrow\Gamma^q(C_{2c})_{12}=1,\quad\Gamma^q(C_{2c})_{22}=0,\quad\Gamma^q(C_{2c})_{11}=0,\quad\Gamma^q(C_{2c})_{21}=1,\\
            \Longrightarrow\Gamma^q(C_{2c})=\left(\begin{matrix}
                0&1\\
                1&0
            \end{matrix}\right).
        \end{gather}
        For $T=C_{2d}$, we need
        \begin{align}
            Q(C_{2d})x=&-z=\Gamma^q(C_{2c})_{12}z+\Gamma^q(C_{2d})_{22}x,\\
            Q(C_{2d})z=&-x=\Gamma^q(C_{2c})_{11}z+\Gamma^q(C_{2d})_{21}x,
        \end{align}
        \begin{gather}
            \Longrightarrow\Gamma^q(C_{2d})_{12}=0,\quad\Gamma^q(C_{2d})_{22}=-1,\quad\Gamma^q(C_{2d})_{11}=0,\quad\Gamma^q(C_{2d})_{21}=-1,\\
            \Longrightarrow\Gamma^q(C_{2c})=\left(\begin{matrix}
                0&-1\\
                -1&0
            \end{matrix}\right).
        \end{gather}
        We find that the $\Gamma^q$ is exactly $\Gamma^5$. Therefore, $x$ is a irreducible tensor operator for irreducible representation $\Gamma^5$ of $D_4$.
        \item[(b)] The Hamiltonian under the interaction is
        \begin{align}
            H=H_0+Ax
        \end{align}
        whose symmetry group is
        \begin{align}
            D_1=\{E,C_{2x}\}
        \end{align}
        $D_1$, with order of $2$, has two classes:
        \begin{align}
            \{E\},\quad\{C_{2x}\},
        \end{align}
        so $D_1$ has two inequivalent irreducible $1$-dimensional representation, one of which is the identity representation, $\Gamma_{D_1}^1$:
        \begin{align}
            \Gamma_{D_1}^1(E)=\Gamma_{D_1}^1(C_{2x})=1,
        \end{align}
        another is
        \begin{align}
            \Gamma_{D_1}^2(E)=\Gamma_{D_1}^2(C_{2x})=-1.
        \end{align}
        \newpage
        From problem $2$, we know that the representation of $D_4$ corresponding to $\psi_{mn}$ and $\psi_{nm}$ is $\Gamma^5$. We write the characters of $\Gamma^5$ with character table of $D_1$ together:
        \begin{table}[h]
            \centering
            \begin{tabular}{c|cc}
             & $\{E\}$ & $\{C_{2x}\}$ \\ \hline
            $\Gamma^5$ & $2$ & $0$ \\
            $\Gamma_{D_1}^1$ & $1$ & $1$ \\
            $\Gamma_{D_1}^2$ & $1$ & $-1$
            \end{tabular}
            \end{table}
            \\We can easily find that
            \begin{align}
                \Gamma^5=\Gamma_{D_1}^1\oplus\Gamma_{D_1}^2.
            \end{align}
            Therefore, the irreducible representations which the allowed final transform as are $\Gamma_{D_1}^1$ and $\Gamma_{D_1}^2$.
    \end{enumerate}
\end{sol}
\end{document}
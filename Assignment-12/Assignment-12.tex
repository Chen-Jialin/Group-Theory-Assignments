% !TEX program = pdflatex
% Group Theory Assignment 12
\documentclass[UTF8,10pt,a4paper]{article}
\usepackage[scheme=plain]{ctex}
\newcommand{\CourseName}{Group Theory}
\newcommand{\CourseCode}{PHYS2102}
\newcommand{\Semester}{Spring, 2020}
\newcommand{\ProjectName}{Assignment 12}
\newcommand{\DueTimeType}{Due Time}
\newcommand{\DueTime}{8:15, June 13, 2020 (Wednesday)}
\newcommand{\StudentName}{陈稼霖}
\newcommand{\StudentID}{45875852}
\usepackage[vmargin=1in,hmargin=.5in]{geometry}
\usepackage{fancyhdr}
\usepackage{lastpage}
\usepackage{calc}
\pagestyle{fancy}
\fancyhf{}
\fancyhead[L]{\CourseName}
\fancyhead[C]{\ProjectName}
\fancyhead[R]{\StudentName}
\fancyfoot[R]{\thepage\ / \pageref{LastPage}}
\setlength\headheight{12pt}
\fancypagestyle{FirstPageStyle}{
    \fancyhf{}
    \fancyhead[L]{\CourseName\\
        \CourseCode\\
        \Semester}
    \fancyhead[C]{{\Huge\bfseries\ProjectName}\\
        \DueTimeType\ : \DueTime}
    \fancyhead[R]{Name : \makebox[\widthof{\StudentID}][s]{\StudentName}\\
        Student ID\@ : \StudentID\\
        Score : \underline{\makebox[\widthof{\StudentID}]{}}}
    \fancyfoot[R]{\thepage\ / \pageref{LastPage}}
    \setlength\headheight{36pt}
}
\usepackage{amsmath,amssymb,amsthm,bm}
\allowdisplaybreaks[4]
\newtheoremstyle{Problem}
{}
{}
{}
{}
{\bfseries}
{.}
{ }
{\thmname{#1}\thmnumber{ #2}\thmnote{ (#3)} Score: \underline{\qquad\qquad}}
\theoremstyle{Problem}
\newtheorem{prob}{Problem}
\newtheoremstyle{Solution}
{}
{}
{}
{}
{\bfseries}
{:}
{ }
{\thmname{#1}}
\makeatletter
\def\@endtheorem{\qed\endtrivlist\@endpefalse}
\makeatother
\theoremstyle{Solution}
\newtheorem*{sol}{Solution}
% \usepackage{graphicx}
\begin{document}
\thispagestyle{FirstPageStyle}
The analytic homomorphic mapping $\phi$ of SU(2) onto SO(3) is given by
\[
    \phi(u)_{jk}=\frac{1}{2}\text{tr}(\sigma_ju\sigma_ku^{-1}),\quad u=\in\text{SU}(2),\quad j,k=1,2,3.
\]
Note that the Pauli matrices $\sigma_1,\sigma_2$, and $\sigma_3$ are also denoted respectively by $\sigma_x,\sigma_y$, and $\sigma_z$.

\[
    \sigma_1=\sigma_x=\left(\begin{matrix}
        0&1\\
        1&0
    \end{matrix}\right),\quad\sigma_2=\sigma_y=\left(\begin{matrix}
        0&-i\\
        i&0
    \end{matrix}\right),\quad\sigma_3=\sigma_z=\left(\begin{matrix}
        1&0\\
        0&-1
    \end{matrix}\right).
\]

\begin{prob}
    Verify explicitly that a rotation about the $Oz$-axis is indeed obtained from the given mapping for $u=e^{-i\theta\sigma_z/2}$.
\end{prob}
\begin{sol}
    The matrix of counterclockwise rotation about the $Oz$-axis through angle $\theta$ is
    \begin{align}
        R=\left(\begin{matrix}
            \cos\theta&-\sin\theta&0\\
            \sin\theta&\cos\theta&0\\
            0&0&1
        \end{matrix}\right).
    \end{align}
    We then verify that the matrix obtained from the given mapping $\phi$ for $u=e^{-i\theta\sigma_z/2}$ is equal to the above rotation matrix.
    \begin{align}
        u=e^{-i\theta\sigma_z/2}=1+\sum_{j=1}^{\infty}\frac{(-i\theta\sigma_z/2)^j}{j!}.
    \end{align}
    Since
    \begin{align}
        \sigma_z^j=\left(\begin{matrix}
            1&0\\
            0&-1
        \end{matrix}\right)^j=\left\{\begin{array}{ll}
            \left(\begin{matrix}
                1&0\\
                0&-1
            \end{matrix}\right),&j\text{ is odd},\\
            \left(\begin{matrix}
                1&0\\
                0&1
            \end{matrix}\right),&j\text{ is even},
        \end{array}\right.
    \end{align}
    we have
    \begin{align}
        \nonumber u=&1+\sum_{j=1}^{\infty}\frac{(-i\theta\sigma_z/2)^j}{j!}\\
        \nonumber=&1+\left(\begin{matrix}
            1&0\\
            0&-1
        \end{matrix}\right)\sum_{j=0}^{\infty}\frac{(-i)^{2j+1}(\theta/2)^{2j+1}}{(2j+1)!}+\left(\begin{matrix}
            1&0\\
            0&1
        \end{matrix}\right)\sum_{j=1}^{\infty}\frac{(-i)^{2j}(\theta/2)^{2j}}{(2j)!}\\
        \nonumber=&1+(-i)\left(\begin{matrix}
            1&0\\
            0&-1
        \end{matrix}\right)\sum_{j=0}^{\infty}\frac{(-1)^j(\theta/2)^{2j+1}}{(2j+1)!}+\left(\begin{matrix}
            1&0\\
            0&1
        \end{matrix}\right)\sum_{j=1}^{\infty}\frac{(-1)^j(\theta/2)^{2j}}{(2j)!}\\
        \nonumber=&1+(-i)\left(\begin{matrix}
            1&0\\
            0&-1
        \end{matrix}\right)\sin\frac{\theta}{2}+\left(\begin{matrix}
            1&0\\
            0&1
        \end{matrix}\right)(\cos\frac{\theta}{2}-1)\\
        =&\left(\begin{matrix}
            \cos\frac{\theta}{2}-i\sin\frac{\theta}{2}&0\\
            0&\cos\frac{\theta}{2}+i\sin\frac{\theta}{2}
        \end{matrix}\right)=\left(\begin{matrix}
            e^{-i\theta/2}&0\\
            0&e^{i\theta/2}
        \end{matrix}\right),
    \end{align}
    and
    \begin{align}
        u^{-1}=\left(\begin{matrix}
            e^{i\theta/2}&0\\
            0&e^{-i\theta/2}
        \end{matrix}\right).
    \end{align}
    Now we calculate the elements of the matrix in SO(3) corresponding to $u$:
    \begin{align}
        \nonumber\phi(u)_{11}=&\frac{1}{2}\text{tr}(\sigma_1u\sigma_1u^{-1})\\
        \nonumber=&\frac{1}{2}\text{tr}\left[\left(\begin{matrix}
            0&1\\
            1&0
        \end{matrix}\right)\left(\begin{matrix}
            e^{-i\theta/2}&0\\
            0&e^{i\theta/2}
        \end{matrix}\right)\left(\begin{matrix}
            0&1\\
            1&0
        \end{matrix}\right)\left(\begin{matrix}
            e^{i\theta/2}&0\\
            0&e^{-i\theta/2}
        \end{matrix}\right)\right]\\
        =&\frac{1}{2}\text{tr}\left(\begin{matrix}
            e^{i\theta}&0\\
            0&e^{-i\theta}
        \end{matrix}\right)=\cos\theta,
    \end{align}
    \begin{align}
        \nonumber\phi(u)_{12}=&\frac{1}{2}\text{tr}(\sigma_1u\sigma_2u^{-1})\\
        \nonumber=&\frac{1}{2}\text{tr}\left[\left(\begin{matrix}
            0&1\\
            1&0
        \end{matrix}\right)\left(\begin{matrix}
            e^{-i\theta/2}&0\\
            0&e^{i\theta/2}
        \end{matrix}\right)\left(\begin{matrix}
            0&-i\\
            i&0
        \end{matrix}\right)\left(\begin{matrix}
            e^{i\theta/2}&0\\
            0&e^{-i\theta/2}
        \end{matrix}\right)\right]\\
        =&\frac{1}{2}\text{tr}\left(\begin{matrix}
            ie^{i\theta}&0\\
            0&-ie^{-i\theta}
        \end{matrix}\right)=-\sin\theta,
    \end{align}
    \begin{align}
        \nonumber\phi(u)_{13}=&\frac{1}{2}\text{tr}(\sigma_1u\sigma_3u^{-1})\\
        \nonumber=&\frac{1}{2}\text{tr}\left[\left(\begin{matrix}
            0&1\\
            1&0
        \end{matrix}\right)\left(\begin{matrix}
            e^{-i\theta/2}&0\\
            0&e^{i\theta/2}
        \end{matrix}\right)\left(\begin{matrix}
            1&0\\
            0&-1
        \end{matrix}\right)\left(\begin{matrix}
            e^{i\theta/2}&0\\
            0&e^{-i\theta/2}
        \end{matrix}\right)\right]\\
        =&\frac{1}{2}\text{tr}\left(\begin{matrix}
            0&-1\\
            1&0
        \end{matrix}\right)=0,
    \end{align}
    \begin{align}
        \nonumber\phi(u)_{21}=&\frac{1}{2}\text{tr}(\sigma_2u\sigma_1u^{-1})\\
        \nonumber=&\frac{1}{2}\text{tr}\left[\left(\begin{matrix}
            0&-i\\
            i&0
        \end{matrix}\right)\left(\begin{matrix}
            e^{-i\theta/2}&0\\
            0&e^{i\theta/2}
        \end{matrix}\right)\left(\begin{matrix}
            0&1\\
            1&0
        \end{matrix}\right)\left(\begin{matrix}
            e^{i\theta/2}&0\\
            0&e^{-i\theta/2}
        \end{matrix}\right)\right]\\
        =&\frac{1}{2}\text{tr}\left(\begin{matrix}
            -ie^{i\theta}&0\\
            0&ie^{-i\theta}
        \end{matrix}\right)=\sin\theta,
    \end{align}
    \begin{align}
        \nonumber\phi(u)_{22}=&\frac{1}{2}\text{tr}(\sigma_2u\sigma_2u^{-1})\\
        \nonumber=&\frac{1}{2}\text{tr}\left[\left(\begin{matrix}
            0&-i\\
            i&0
        \end{matrix}\right)\left(\begin{matrix}
            e^{-i\theta/2}&0\\
            0&e^{i\theta/2}
        \end{matrix}\right)\left(\begin{matrix}
            0&-i\\
            i&0
        \end{matrix}\right)\left(\begin{matrix}
            e^{i\theta/2}&0\\
            0&e^{-i\theta/2}
        \end{matrix}\right)\right]\\
        =&\frac{1}{2}\text{tr}\left(\begin{matrix}
            e^{i\theta}&0\\
            0&e^{-i\theta}
        \end{matrix}\right)=\cos\theta,
    \end{align}
    \begin{align}
        \nonumber\phi(u)_{23}=&\frac{1}{2}\text{tr}(\sigma_2u\sigma_3u^{-1})\\
        \nonumber=&\frac{1}{2}\text{tr}\left[\left(\begin{matrix}
            0&-i\\
            i&0
        \end{matrix}\right)\left(\begin{matrix}
            e^{-i\theta/2}&0\\
            0&e^{i\theta/2}
        \end{matrix}\right)\left(\begin{matrix}
            1&0\\
            0&-1
        \end{matrix}\right)\left(\begin{matrix}
            e^{i\theta/2}&0\\
            0&e^{-i\theta/2}
        \end{matrix}\right)\right]\\
        =&\frac{1}{2}\text{tr}\left(\begin{matrix}
            0&i\\
            i&0
        \end{matrix}\right)=0,
    \end{align}
    \begin{align}
        \nonumber\phi(u)_{31}=&\frac{1}{2}\text{tr}(\sigma_3u\sigma_1u^{-1})\\
        \nonumber=&\frac{1}{2}\text{tr}\left[\left(\begin{matrix}
            1&0\\
            0&-1
        \end{matrix}\right)\left(\begin{matrix}
            e^{-i\theta/2}&0\\
            0&e^{i\theta/2}
        \end{matrix}\right)\left(\begin{matrix}
            0&1\\
            1&0
        \end{matrix}\right)\left(\begin{matrix}
            e^{i\theta/2}&0\\
            0&e^{-i\theta/2}
        \end{matrix}\right)\right]\\
        =&\frac{1}{2}\text{tr}\left(\begin{matrix}
            0&e^{-i\theta}\\
            -e^{i\theta}&0
        \end{matrix}\right)=0,
    \end{align}
    \begin{align}
        \nonumber\phi(u)_{32}=&\frac{1}{2}\text{tr}(\sigma_3u\sigma_2u^{-1})\\
        \nonumber=&\frac{1}{2}\text{tr}\left[\left(\begin{matrix}
            1&0\\
            0&-1
        \end{matrix}\right)\left(\begin{matrix}
            e^{-i\theta/2}&0\\
            0&e^{i\theta/2}
        \end{matrix}\right)\left(\begin{matrix}
            0&-i\\
            i&0
        \end{matrix}\right)\left(\begin{matrix}
            e^{i\theta/2}&0\\
            0&e^{-i\theta/2}
        \end{matrix}\right)\right]\\
        =&\frac{1}{2}\text{tr}\left(\begin{matrix}
            0&-ie^{-i\theta}\\
            ie^{i\theta}&0
        \end{matrix}\right)=0,
    \end{align}
    \begin{align}
        \nonumber\phi(u)_{33}=&\frac{1}{2}\text{tr}(\sigma_3u\sigma_3u^{-1})\\
        \nonumber=&\frac{1}{2}\text{tr}\left[\left(\begin{matrix}
            1&0\\
            0&-1
        \end{matrix}\right)\left(\begin{matrix}
            e^{-i\theta/2}&0\\
            0&e^{i\theta/2}
        \end{matrix}\right)\left(\begin{matrix}
            1&0\\
            0&-1
        \end{matrix}\right)\left(\begin{matrix}
            e^{i\theta/2}&0\\
            0&e^{-i\theta/2}
        \end{matrix}\right)\right]\\
        =&\frac{1}{2}\text{tr}\left(\begin{matrix}
            1&0\\
            0&1
        \end{matrix}\right)=1.
    \end{align}
    The matrix in SO(3) corresponding to $u$ is
    \begin{align}
        \phi(u)=\left(\begin{matrix}
            \cos\theta&-\sin\theta&0\\
            \sin\theta&\cos\theta&0\\
            0&0&1
        \end{matrix}\right),
    \end{align}
    which is exactly the rotation matrix we write at the beginning. Therefore, a rotation about the $Oz$-axis is indeed obtained from the given mapping for $u=e^{-i\theta\sigma_z/2}$.
\end{sol}

\begin{prob}
    Verify explicitly that a rotation about the $Oy$-axis is indeed obtained from the given mapping for $u=e^{-i\theta\sigma_y/2}$.
\end{prob}
\begin{sol}
    The matrix of counterclockwise rotation about the $Oy$-axis through angle $\theta$ is
    \begin{align}
        R=\left(\begin{matrix}
            \cos\theta&0&\sin\theta\\
            0&1&0\\
            -\sin\theta&0&\cos\theta
        \end{matrix}\right).
    \end{align}
    We then verify that the matrix obtained from the given mapping $\phi$ for $u=e^{-i\theta\sigma_y/2}$ is equal to the above rotation matrix.
    \begin{align}
        u=e^{-i\theta\sigma_y/2}=1+\sum_{j=1}^{\infty}\frac{(-i\theta\sigma_y/2)^j}{j!}.
    \end{align}
    Since
    \begin{align}
        \sigma_y^j=\left\{\begin{array}{ll}
            \left(\begin{matrix}
                0&-i\\
                i&0
            \end{matrix}\right),&j\text{ is odd},\\
            \left(\begin{matrix}
                1&0\\
                0&1
            \end{matrix}\right),&j\text{ is even},
        \end{array}\right.
    \end{align}
    we have
    \begin{align}
        \nonumber u=&1+\sum_{j=1}\frac{(-i\theta\sigma_y/2)^j}{j!}\\
        \nonumber=&1+\left(\begin{matrix}
            0&-i\\
            i&0
        \end{matrix}\right)\sum_{j=0}^{\infty}\frac{(-i)^{2j+1}(\theta/2)^{2j+1}}{(2j+1)}+\left(\begin{matrix}
            1&0\\
            0&1
        \end{matrix}\right)\sum_{j=1}^{\infty}\frac{(-i)^{2j}(\theta/2)^{2j}}{(2j)!}\\
        \nonumber=&1+(-i)\left(\begin{matrix}
            0&-i\\
            i&0
        \end{matrix}\right)\sum_{j=0}^{\infty}\frac{(-1)^j(\theta/2)^{2j+1}}{(2j+1)}+\left(\begin{matrix}
            1&0\\
            0&1
        \end{matrix}\right)\sum_{j=1}^{\infty}\frac{(-1)^j(\theta/2)^{2j}}{(2j)!}\\
        \nonumber=&1+(-i)\left(\begin{matrix}
            0&-i\\
            i&0
        \end{matrix}\right)\sin\frac{\theta}{2}+\left(\begin{matrix}
            1&0\\
            0&1
        \end{matrix}\right)(\cos\frac{\theta}{2}-1)\\
        =&\left(\begin{matrix}
            \cos\frac{\theta}{2}&-\sin\frac{\theta}{2}\\
            \sin\frac{\theta}{2}&\cos\frac{\theta}{2}
        \end{matrix}\right),
    \end{align}
    and
    \begin{align}
        u^{-1}=\left(\begin{matrix}
            \cos\frac{\theta}{2}&\sin\frac{\theta}{2}\\
            -\sin\frac{\theta}{2}&\cos\frac{\theta}{2}
        \end{matrix}\right).
    \end{align}
    Now we calculate the elements of the matrix in SO(3) corresponding to $u$:
    \begin{align}
        \nonumber\phi(u)_{11}=&\frac{1}{2}\text{tr}(\sigma_1u\sigma_1u^{-1})\\
        \nonumber=&\frac{1}{2}\text{tr}\left[\left(\begin{matrix}
            0&1\\
            1&0
        \end{matrix}\right)\left(\begin{matrix}
            \cos\frac{\theta}{2}&-\sin\frac{\theta}{2}\\
            \sin\frac{\theta}{2}&\cos\frac{\theta}{2}
        \end{matrix}\right)\left(\begin{matrix}
            0&1\\
            1&0
        \end{matrix}\right)\left(\begin{matrix}
            \cos\frac{\theta}{2}&\sin\frac{\theta}{2}\\
            -\sin\frac{\theta}{2}&\cos\frac{\theta}{2}
        \end{matrix}\right)\right]\\
        =&\frac{1}{2}\text{tr}\left(\begin{matrix}
            \cos^2\frac{\theta}{2}-\sin^2\frac{\theta}{2}&2\sin\frac{\theta}{2}\cos\frac{\theta}{2}\\
            -2\sin\frac{\theta}{2}\cos\frac{\theta}{2}&\cos^2\frac{\theta}{2}-\sin^2\frac{\theta}{2}
        \end{matrix}\right)=\frac{1}{2}\text{tr}\left(\begin{matrix}
            \cos\theta&\sin\theta\\
            -\sin\theta&\cos\theta
        \end{matrix}\right)=\cos\theta,
    \end{align}
    \begin{align}
        \nonumber\phi(u)_{12}=&\frac{1}{2}\text{tr}(\sigma_1u\sigma_2u^{-1})\\
        \nonumber=&\frac{1}{2}\text{tr}\left[\left(\begin{matrix}
            0&1\\
            1&0
        \end{matrix}\right)\left(\begin{matrix}
            \cos\frac{\theta}{2}&-\sin\frac{\theta}{2}\\
            \sin\frac{\theta}{2}&\cos\frac{\theta}{2}
        \end{matrix}\right)\left(\begin{matrix}
            0&-i\\
            i&0
        \end{matrix}\right)\left(\begin{matrix}
            \cos\frac{\theta}{2}&\sin\frac{\theta}{2}\\
            -\sin\frac{\theta}{2}&\cos\frac{\theta}{2}
        \end{matrix}\right)\right]\\
        =&\frac{1}{2}\text{tr}\left(\begin{matrix}
            i&0\\
            0&-i
        \end{matrix}\right)=0,
    \end{align}
    \begin{align}
        \nonumber\phi(u)_{13}=&\frac{1}{2}\text{tr}(\sigma_1u\sigma_3u^{-1})\\
        \nonumber=&\frac{1}{2}\text{tr}\left[\left(\begin{matrix}
            0&1\\
            1&0
        \end{matrix}\right)\left(\begin{matrix}
            \cos\frac{\theta}{2}&-\sin\frac{\theta}{2}\\
            \sin\frac{\theta}{2}&\cos\frac{\theta}{2}
        \end{matrix}\right)\left(\begin{matrix}
            1&0\\
            0&-1
        \end{matrix}\right)\left(\begin{matrix}
            \cos\frac{\theta}{2}&\sin\frac{\theta}{2}\\
            -\sin\frac{\theta}{2}&\cos\frac{\theta}{2}
        \end{matrix}\right)\right]\\
        =&\frac{1}{2}\text{tr}\left(\begin{matrix}
            2\sin\frac{\theta}{2}\cos\frac{\theta}{2}&\sin^2\frac{\theta}{2}-\cos^2\frac{\theta}{2}\\
            \cos^2\frac{\theta}{2}-\sin^2\frac{\theta}{2}&2\sin\frac{\theta}{2}\cos\frac{\theta}{2}
        \end{matrix}\right)=\frac{1}{2}\text{tr}\left(\begin{matrix}
            \sin\theta&-\cos\theta\\
            \cos\theta&\sin\theta
        \end{matrix}\right)=\sin\theta,
    \end{align}
    \begin{align}
        \nonumber\phi(u)_{21}=&\frac{1}{2}\text{tr}(\sigma_2u\sigma_1u^{-1})\\
        \nonumber=&\frac{1}{2}\text{tr}\left[\left(\begin{matrix}
            0&-i\\
            i&0
        \end{matrix}\right)\left(\begin{matrix}
            \cos\frac{\theta}{2}&-\sin\frac{\theta}{2}\\
            \sin\frac{\theta}{2}&\cos\frac{\theta}{2}
        \end{matrix}\right)\left(\begin{matrix}
            0&1\\
            1&0
        \end{matrix}\right)\left(\begin{matrix}
            \cos\frac{\theta}{2}&\sin\frac{\theta}{2}\\
            -\sin\frac{\theta}{2}&\cos\frac{\theta}{2}
        \end{matrix}\right)\right]\\
        =&\frac{1}{2}\text{tr}\left(\begin{matrix}
            i(\sin^2\frac{\theta}{2}-i\cos^2\frac{\theta}{2})&-2i\sin\frac{\theta}{2}\cos\frac{\theta}{2}\\
            -2i\sin\frac{\theta}{2}\cos\frac{\theta}{2}&i(\cos^2\frac{\theta}{2}-\sin^2\frac{\theta}{2})
        \end{matrix}\right)=\frac{1}{2}\text{tr}\left(\begin{matrix}
            -i\cos\theta&-i\sin\theta\\
            -i\sin\theta&i\cos\theta
        \end{matrix}\right)=0,
    \end{align}
    \begin{align}
        \nonumber\phi(u)_{22}=&\frac{1}{2}\text{tr}(\sigma_2u\sigma_2u^{-1})\\
        \nonumber=&\frac{1}{2}\text{tr}\left[\left(\begin{matrix}
            0&-i\\
            i&0
        \end{matrix}\right)\left(\begin{matrix}
            \cos\frac{\theta}{2}&-\sin\frac{\theta}{2}\\
            \sin\frac{\theta}{2}&\cos\frac{\theta}{2}
        \end{matrix}\right)\left(\begin{matrix}
            0&-i\\
            i&0
        \end{matrix}\right)\left(\begin{matrix}
            \cos\frac{\theta}{2}&\sin\frac{\theta}{2}\\
            -\sin\frac{\theta}{2}&\cos\frac{\theta}{2}
        \end{matrix}\right)\right]\\
        =&\frac{1}{2}\text{tr}\left(\begin{matrix}
            1&0\\
            0&1
        \end{matrix}\right)=1,
    \end{align}
    \begin{align}
        \nonumber\phi(u)_{23}=&\frac{1}{2}\text{tr}(\sigma_2u\sigma_3u^{-1})\\
        \nonumber=&\frac{1}{2}\text{tr}\left[\left(\begin{matrix}
            0&-i\\
            i&0
        \end{matrix}\right)\left(\begin{matrix}
            \cos\frac{\theta}{2}&-\sin\frac{\theta}{2}\\
            \sin\frac{\theta}{2}&\cos\frac{\theta}{2}
        \end{matrix}\right)\left(\begin{matrix}
            1&0\\
            0&-1
        \end{matrix}\right)\left(\begin{matrix}
            \cos\frac{\theta}{2}&\sin\frac{\theta}{2}\\
            -\sin\frac{\theta}{2}&\cos\frac{\theta}{2}
        \end{matrix}\right)\right]\\
        =&\frac{1}{2}\text{tr}\left(\begin{matrix}
            -2i\sin\frac{\theta}{2}\cos\frac{\theta}{2}&i\cos^2\frac{\theta}{2}-i\sin^2\frac{\theta}{2}\\
            i\cos^2\frac{\theta}{2}-i\sin^2\frac{\theta}{2}&2i\sin\frac{\theta}{2}\cos\frac{\theta}{2}
        \end{matrix}\right)=\frac{1}{2}\left(\begin{matrix}
            -i\sin\theta&i\cos\theta\\
            i\cos\theta&i\sin\theta
        \end{matrix}\right)=0,
    \end{align}
    \begin{align}
        \nonumber\phi(u)_{31}=&\frac{1}{2}\text{tr}(\sigma_3u\sigma_1u^{-1})\\
        \nonumber=&\frac{1}{2}\text{tr}\left[\left(\begin{matrix}
            1&0\\
            0&-1
        \end{matrix}\right)\left(\begin{matrix}
            \cos\frac{\theta}{2}&-\sin\frac{\theta}{2}\\
            \sin\frac{\theta}{2}&\cos\frac{\theta}{2}
        \end{matrix}\right)\left(\begin{matrix}
            0&1\\
            1&0
        \end{matrix}\right)\left(\begin{matrix}
            \cos\frac{\theta}{2}&\sin\frac{\theta}{2}\\
            -\sin\frac{\theta}{2}&\cos\frac{\theta}{2}
        \end{matrix}\right)\right]\\
        =&\frac{1}{2}\text{tr}\left(\begin{matrix}
            -2\sin\frac{\theta}{2}\cos\frac{\theta}{2}&\cos^2\frac{\theta}{2}-\sin^2\frac{\theta}{2}\\
            \sin^2\frac{\theta}{2}-\cos^2\frac{\theta}{2}&-2\sin\frac{\theta}{2}\cos\frac{\theta}{2}
        \end{matrix}\right)=\frac{1}{2}\text{tr}\left(\begin{matrix}
            -\sin\theta&\cos\theta\\
            -\cos\theta&-\sin\theta
        \end{matrix}\right)=-\sin\theta,
    \end{align}
    \begin{align}
        \nonumber\phi(u)_{32}=&\frac{1}{2}\text{tr}(\sigma_3u\sigma_2u^{-1})\\
        \nonumber=&\frac{1}{2}\text{tr}\left[\left(\begin{matrix}
            1&0\\
            0&-1
        \end{matrix}\right)\left(\begin{matrix}
            \cos\frac{\theta}{2}&-\sin\frac{\theta}{2}\\
            \sin\frac{\theta}{2}&\cos\frac{\theta}{2}
        \end{matrix}\right)\left(\begin{matrix}
            0&-i\\
            i&0
        \end{matrix}\right)\left(\begin{matrix}
            \cos\frac{\theta}{2}&\sin\frac{\theta}{2}\\
            -\sin\frac{\theta}{2}&\cos\frac{\theta}{2}
        \end{matrix}\right)\right]\\
        =&\frac{1}{2}\text{tr}\left(\begin{matrix}
            0&-i\\
            -i&0
        \end{matrix}\right)=0,
    \end{align}
    \begin{align}
        \nonumber\phi(u)_{33}=&\frac{1}{2}\text{tr}(\sigma_3u\sigma_3u^{-1})\\
        \nonumber=&\frac{1}{2}\text{tr}\left[\left(\begin{matrix}
            1&0\\
            0&-1
        \end{matrix}\right)\left(\begin{matrix}
            \cos\frac{\theta}{2}&-\sin\frac{\theta}{2}\\
            \sin\frac{\theta}{2}&\cos\frac{\theta}{2}
        \end{matrix}\right)\left(\begin{matrix}
            1&0\\
            0&-1
        \end{matrix}\right)\left(\begin{matrix}
            \cos\frac{\theta}{2}&\sin\frac{\theta}{2}\\
            -\sin\frac{\theta}{2}&\cos\frac{\theta}{2}
        \end{matrix}\right)\right]\\
        =&\frac{1}{2}\text{tr}\left(\begin{matrix}
            \cos^2\frac{\theta}{2}-\sin^2\frac{\theta}{2}&2\sin\frac{\theta}{2}\cos\frac{\theta}{2}\\
            -2\sin\frac{\theta}{2}\cos\frac{\theta}{2}&\cos^2\frac{\theta}{2}-\sin^2\frac{\theta}{2}
        \end{matrix}\right)=\frac{1}{2}\text{tr}\left(\begin{matrix}
            \cos\theta&2\sin\theta\\
            -2\sin\theta&\cos\theta
        \end{matrix}\right)=\cos\theta.
    \end{align}
    The matrix in SO(3) corresponding to $u$ is
    \begin{align}
        \phi(u)=\left(\begin{matrix}
            \cos\theta&0&\sin\theta\\
            0&1&0\\
            -\sin\theta&0&\cos\theta
        \end{matrix}\right),
    \end{align}
    which is exactly the rotation matrix we write at the beginning. Therefore, a rotation about the $Oy$-axis is indeed obtained from the given mapping for $u=e^{-i\theta\sigma_y/2}$.
\end{sol}

\begin{prob}
    Show that the analytic isomorphic mapping of the real Lie algebra su(2) onto the real algebra so(3) is given by
    \[
        \psi(a)_{jk}=\frac{1}{2}\text{tr}(\sigma_j[a,\sigma_k]),\quad a\in\text{su}(2),\quad j,k=1,2,3.
    \]
\end{prob}
\begin{sol}
    The analytic isomorphic mapping of the real Lie algebra su(2) onto the real algebra so(3) is
    \begin{align}
        \nonumber\psi(a)_{jk}=&\left.\frac{d\phi(e^{ta})}{dt}\right\rvert_{t=0}\\
        \nonumber=&\left.\frac{d\left\{\frac{1}{2}\text{tr}[\sigma_je^{ta}\sigma_k(e^{ta})^{-1}]\right\}}{dt}\right\rvert_{t=0}\\
        \nonumber=&\lim_{t\rightarrow 0}\frac{\frac{1}{2}\text{tr}[\sigma_je^{ta}\sigma_k(e^{ta})^{-1}]-\frac{1}{2}\text{tr}[\sigma_j\sigma_k]}{t}\\
        \nonumber&(\text{using }e^{ta}b[e^{ta}]^{-1}=b+t[a,b]+\frac{1}{2}[a,[a,b]]+\cdots)\\
        \nonumber=&\lim_{t\rightarrow 0}\frac{\frac{1}{2}\text{tr}\left\{\sigma_j\left(\sigma_k+t[a,\sigma_k]+\frac{1}{2}t^2[a,[a,\sigma_k]]+\cdots\right)\right\}-\frac{1}{2}\text{tr}\left\{\sigma_j\sigma_k\right\}}{t}\\
        =&\frac{1}{2}\text{tr}(\sigma_j[a,\sigma_k]),
    \end{align}
    for $a\in\text{su}(2),\,j,k=1,2,3$.
\end{sol}

\begin{prob}
    Without using the explicit matrix representations of the Pauli matrices, show that $\psi(a_p)_{jk}=\epsilon_{pjk}$ for $a_p=i\sigma_p/2$ with $p=1,2,3$.\\
    \textbf{Hint:} $[\sigma_p,\sigma_k]=2i\sum_{l=1}^3\epsilon_{pkl}\sigma_l$.
\end{prob}
\begin{sol}
    Using the conclusion we obtained in the latter problem, we have
    \begin{align}
        \nonumber\psi(a_p)_{jk}=&\psi(i\sigma_p/2)=\frac{1}{2}\text{tr}(\sigma_j[i\sigma_p/2,\sigma_k])=\frac{1}{2}\text{tr}\left(\frac{i}{2}\sigma_j[\sigma_p,\sigma_k]\right)\\
        \nonumber=&\frac{1}{2}\text{tr}\left(-\sigma_j\sum_{l=1}^3\epsilon_{pkl}\sigma_l\right)\\
        \nonumber&(\text{using }\sigma_a\sigma_b=\delta_{ab}I+i\sum_{c=1}^3\epsilon_{abc}\sigma_c,\text{ where }I\text{ is the identity matrix})\\
        \nonumber=&\frac{1}{2}\text{tr}\left(-\sum_{l=1}^3\epsilon_{pkl}\left(\delta_{jl}I+i\sum_{m=1}^3\epsilon_{jlm}\sigma_m\right)\right)\\
        \nonumber&(\text{using }\text{tr}(I)=2,\text{ and }\text{tr}\sigma_m=0,\quad m=1,2,3)\\
        \nonumber=&-\sum_{l=1}^3\epsilon_{pkl}\delta_{lj}\\
        \nonumber=&-\epsilon_{pkj}\\
        =&\epsilon_{pjk}.
    \end{align}
\end{sol}

\begin{prob}
    Using the definition of $\epsilon_{pjk}$, show that
    \[
        \psi(a_1)=\left(\begin{matrix}
            0&0&0\\
            0&0&1\\
            0&-1&0
        \end{matrix}\right),\quad\psi(a_2)=\left(\begin{matrix}
            0&0&-1\\
            0&0&0\\
            1&0&0
        \end{matrix}\right),\quad\psi(a_3)=\left(\begin{matrix}
            0&1&0\\
            -1&0&0\\
            0&0&0
        \end{matrix}\right).
    \]
\end{prob}
\begin{sol}
    Using the conclusion we obtained in the latter problem and the definition of $\epsilon_{pjk}$, the matrix elements of $\psi(a_1)$ are
    \begin{align}
        \psi(a_1)_{11}=&\epsilon_{111}=0,\\
        \psi(a_1)_{12}=&\epsilon_{112}=0,\\
        \psi(a_1)_{13}=&\epsilon_{113}=0,\\
        \psi(a_1)_{21}=&\epsilon_{121}=0,\\
        \psi(a_1)_{22}=&\epsilon_{122}=0,\\
        \psi(a_1)_{23}=&\epsilon_{123}=1,\\
        \psi(a_1)_{31}=&\epsilon_{131}=0,\\
        \psi(a_1)_{32}=&\epsilon_{132}=-1,\\
        \psi(a_1)_{33}=&\epsilon_{133}=0,\\
    \end{align}
    so
    \begin{align}
        \psi(a_1)=\left(\begin{matrix}
            0&0&0\\
            0&0&1\\
            0&-1&0
        \end{matrix}\right).
    \end{align}
    The matrix elements of $\psi(a_2)$ are
    \begin{align}
        \psi(a_2)_{11}=&\epsilon_{211}=0,\\
        \psi(a_2)_{12}=&\epsilon_{212}=0,\\
        \psi(a_2)_{13}=&\epsilon_{213}=-1,\\
        \psi(a_2)_{21}=&\epsilon_{221}=0,\\
        \psi(a_2)_{22}=&\epsilon_{222}=0,\\
        \psi(a_2)_{23}=&\epsilon_{223}=0,\\
        \psi(a_2)_{31}=&\epsilon_{231}=1,\\
        \psi(a_2)_{32}=&\epsilon_{232}=0,\\
        \psi(a_2)_{33}=&\epsilon_{233}=0,\\
    \end{align}
    so
    \begin{align}
        \psi(a_2)=\left(\begin{matrix}
            0&0&-1\\
            0&0&0\\
            1&0&0
        \end{matrix}\right).
    \end{align}
    The matrix elements of $\psi(a_3)$ are
    \begin{align}
        \psi(a_3)_{11}=&\epsilon_{311}=0,\\
        \psi(a_3)_{12}=&\epsilon_{312}=1,\\
        \psi(a_3)_{13}=&\epsilon_{313}=0,\\
        \psi(a_3)_{21}=&\epsilon_{321}=-1,\\
        \psi(a_3)_{22}=&\epsilon_{322}=0,\\
        \psi(a_3)_{23}=&\epsilon_{323}=0,\\
        \psi(a_3)_{31}=&\epsilon_{331}=0,\\
        \psi(a_3)_{32}=&\epsilon_{332}=0,\\
        \psi(a_3)_{33}=&\epsilon_{333}=0,\\
    \end{align}
    so
    \begin{align}
        \psi(a_3)=\left(\begin{matrix}
            0&1&0\\
            -1&0&0\\
            0&0&0
        \end{matrix}\right).
    \end{align}
\end{sol}
\end{document}
% !TEX program = pdflatex
% !TEX options = -synctex=1 -interaction=nonstopmode -file-line-error "%DOC%"
% Group Theory Assignment 06
\documentclass[UTF8,10pt,a4paper]{article}
\usepackage[scheme=plain]{ctex}
\newcommand{\CourseName}{Group Theory}
\newcommand{\CourseCode}{PHYS2102}
\newcommand{\Semester}{Spring, 2020}
\newcommand{\ProjectName}{Assignment 06}
\newcommand{\DueTimeType}{Due Time}
\newcommand{\DueTime}{8:15, April 15, 2020 (Wednesday)}
\newcommand{\StudentName}{陈稼霖}
\newcommand{\StudentID}{45875852}
\usepackage[vmargin=1in,hmargin=.5in]{geometry}
\usepackage{fancyhdr}
\usepackage{lastpage}
\usepackage{calc}
\pagestyle{fancy}
\fancyhf{}
\fancyhead[L]{\CourseName}
\fancyhead[C]{\ProjectName}
\fancyhead[R]{\StudentName}
\fancyfoot[R]{\thepage\ / \pageref{LastPage}}
\setlength\headheight{12pt}
\fancypagestyle{FirstPageStyle}{
    \fancyhf{}
    \fancyhead[L]{\CourseName\\
        \CourseCode\\
        \Semester}
    \fancyhead[C]{{\Huge\bfseries\ProjectName}\\
        \DueTimeType\ : \DueTime}
    \fancyhead[R]{Name : \makebox[\widthof{\StudentID}][s]{\StudentName}\\
        Student ID\@ : \StudentID\\
        Score : \underline{\makebox[\widthof{\StudentID}]{}}}
    \fancyfoot[R]{\thepage\ / \pageref{LastPage}}
    \setlength\headheight{36pt}
}
\usepackage{amsmath,amssymb,amsthm,bm}
\allowdisplaybreaks[4]
\newtheoremstyle{Problem}
{}
{}
{}
{}
{\bfseries}
{.}
{ }
{\thmname{#1}\thmnumber{ #2}\thmnote{ (#3)} Score: \underline{\qquad\qquad}}
\theoremstyle{Problem}
\newtheorem{prob}{Problem}
\newtheoremstyle{Solution}
{}
{}
{}
{}
{\bfseries}
{:}
{ }
{\thmname{#1}}
\makeatletter
\def\@endtheorem{\qed\endtrivlist\@endpefalse}
\makeatother
\theoremstyle{Solution}
\newtheorem*{sol}{Solution}
% \usepackage{graphicx}
\begin{document}
\thispagestyle{FirstPageStyle}
\begin{prob}
    The basis elements of the real Lie algebra $L=\text{so}(3)$ are given by
    \[
        a_1=\left(\begin{matrix}
            0&0&0\\
            0&0&1\\
            0&-1&0
        \end{matrix}\right),\quad a_2=\left(\begin{matrix}
            0&0&-1\\
            0&0&0\\
            1&0&0
        \end{matrix}\right),\quad a_3=\left(\begin{matrix}
            0&1&0\\
            -1&0&0\\
            0&0&0
        \end{matrix}\right).
    \]
    Show explicitly that these basis elements possess the following properties.
    \begin{enumerate}
        \item[(a)] The basis elements $a_1$, $a_2$, and $a_3$ obey the commutation relations
        \begin{align*}
            [a_1,a_2]=&a_1a_2-a_2a_1=-a_3,\\
            [a_2,a_3]=&a_2a_3-a_3a_2=-a_1,\\
            [a_3,a_1]=&a_3a_1-a_1a_3=-a_2.
        \end{align*}
        \item[(b)] The basis elements $a_1$, $a_2$, and $a_3$ are anti-Hermitian,
        \[
            a_1^{\dagger}=-a_1,\quad a_2^{\dagger}=-a_2,\quad a_3^{\dagger}=-a_3.
        \]
    \end{enumerate}
\end{prob}
\begin{sol}
    \begin{enumerate}
        \item[(a)] The basis elements $a_1$, $a_2$, and $a_3$ obey the commutation relations:
        \begin{align}
            \nonumber[a_1,a_2]=&a_1a_2-a_2a_1=\left(\begin{matrix}
                0&0&0\\
                0&0&1\\
                0&-1&0
            \end{matrix}\right)\left(\begin{matrix}
                0&0&-1\\
                0&0&0\\
                1&0&0
            \end{matrix}\right)-\left(\begin{matrix}
                0&0&-1\\
                0&0&0\\
                1&0&0
            \end{matrix}\right)\left(\begin{matrix}
                0&0&0\\
                0&0&1\\
                0&-1&0
            \end{matrix}\right)\\
            =&\left(\begin{matrix}
                0&0&0\\
                1&0&0\\
                0&0&0
            \end{matrix}\right)-\left(\begin{matrix}
                0&1&0\\
                0&0&0\\
                0&0&0
            \end{matrix}\right)=\left(\begin{matrix}
                0&-1&0\\
                1&0&0\\
                0&0&0
            \end{matrix}\right)=-a_3,\\
            \nonumber[a_2,a_3]=&a_2a_3-a_3a_2=\left(\begin{matrix}
                0&0&-1\\
                0&0&0\\
                1&0&0
            \end{matrix}\right)\left(\begin{matrix}
                0&1&0\\
                -1&0&0\\
                0&0&0
            \end{matrix}\right)-\left(\begin{matrix}
                0&1&0\\
                -1&0&0\\
                0&0&0
            \end{matrix}\right)\left(\begin{matrix}
                0&0&-1\\
                0&0&0\\
                1&0&0
            \end{matrix}\right)\\
            =&\left(\begin{matrix}
                0&0&0\\
                0&0&0\\
                0&1&0
            \end{matrix}\right)-\left(\begin{matrix}
                0&0&0\\
                0&0&1\\
                0&0&0
            \end{matrix}\right)=\left(\begin{matrix}
                0&0&0\\
                0&0&-1\\
                0&1&0
            \end{matrix}\right)=-a_1,\\
            \nonumber[a_3,a_1]=&a_3a_1-a_1a_3=\left(\begin{matrix}
                0&1&0\\
                -1&0&0\\
                0&0&0
            \end{matrix}\right)\left(\begin{matrix}
                0&0&0\\
                0&0&1\\
                0&-1&0
            \end{matrix}\right)-\left(\begin{matrix}
                0&0&0\\
                0&0&1\\
                0&-1&0
            \end{matrix}\right)\left(\begin{matrix}
                0&1&0\\
                -1&0&0\\
                0&0&0
            \end{matrix}\right)\\
            =&\left(\begin{matrix}
                0&0&1\\
                0&0&0\\
                0&0&0
            \end{matrix}\right)-\left(\begin{matrix}
                0&0&0\\
                0&0&0\\
                1&0&0
            \end{matrix}\right)=\left(\begin{matrix}
                0&0&1\\
                0&0&0\\
                -1&0&0
            \end{matrix}\right)=-a_2.
        \end{align}
        \item[(b)] The basis elements $a_1$, $a_2$, and $a_3$ are anti-Hermitian:
        \begin{align}
            a_1^{\ddagger}=&\left(\begin{matrix}
                0&0&0\\
                0&0&-1\\
                0&1&0
            \end{matrix}\right)=-a_1,\\
            a_2^{\ddagger}=&\left(\begin{matrix}
                0&0&1\\
                0&0&0\\
                -1&0&0
            \end{matrix}\right)=-a_2,\\
            a_3^{\ddagger}=&\left(\begin{matrix}
                0&-1&0\\
                1&0&0\\
                0&0&0
            \end{matrix}\right)=-a_3.
        \end{align}
    \end{enumerate}
\end{sol}

\begin{prob}
    The scalar transformation operators $Q(a_1)$, $Q(a_2)$, and $Q(a_3)$ for the real Lie algebra $\text{so}(3)$ are found to be given by
    \[
        Q(a_1)=y\frac{\partial}{\partial z}-z\frac{\partial}{\partial y},\quad Q(a_2)=z\frac{\partial}{\partial x}-x\frac{\partial}{\partial z},\quad Q(a_3)=x\frac{\partial}{\partial y}-y\frac{\partial}{\partial x}.
    \]
    Show that $[Q(a_1),Q(a_2)]=-Q(a_3)$, $[Q(a_2),Q(a_3)]=-Q(a_1)$, and $[Q(a_3),Q(a_1)]=-Q(a_2)$.
\end{prob}
\begin{sol}
    Applying $[Q(a_1),Q(a_2)]$, $[Q(a_2),Q(a_3)]$, and $[Q(a_3),Q(a_1)]$ to an arbitrary function $f(\vec{r})$ of $\vec{r}$, we get
    \begin{align}
        \nonumber[Q(a_1),Q(a_2)]f(\vec{r})=&\left(y\frac{\partial}{\partial z}-z\frac{\partial}{\partial y}\right)\left(z\frac{\partial f}{\partial x}-x\frac{\partial f}{\partial z}\right)-\left(z\frac{\partial}{\partial x}-x\frac{\partial}{\partial z}\right)\left(y\frac{\partial f}{\partial z}-z\frac{\partial f}{\partial y}\right)\\
        \nonumber=&y\frac{\partial f}{\partial x}+yz\frac{\partial^2f}{\partial z\partial x}-z^2\frac{\partial^2f}{\partial y\partial x}-yx\frac{\partial^2f}{\partial z^2}+zx\frac{\partial^2f}{\partial y\partial z}\\
        \nonumber&-zy\frac{\partial^2f}{\partial x\partial z}+xy\frac{\partial^2f}{\partial z^2}+z^2\frac{\partial^2f}{\partial x\partial y}-x\frac{\partial f}{\partial y}-xz\frac{\partial^2f}{\partial z\partial y}\\
        =&y\frac{\partial f}{\partial x}-x\frac{\partial f}{\partial y}=\left(y\frac{\partial}{\partial x}-x\frac{\partial}{\partial y}\right)f(\vec{r})=-Q(a_3)f(\vec{r}).\\
        \nonumber[Q(a_2),Q(a_3)]f(\vec{r})=&\left(z\frac{\partial}{\partial x}-x\frac{\partial}{\partial z}\right)\left(x\frac{\partial f}{\partial y}-y\frac{\partial f}{\partial x}\right)-\left(x\frac{\partial}{\partial y}-y\frac{\partial}{\partial x}\right)\left(z\frac{\partial f}{\partial x}-x\frac{\partial f}{\partial z}\right)\\
        \nonumber=&z\frac{\partial f}{\partial y}+zx\frac{\partial^2f}{\partial x\partial y}-x^2\frac{\partial^2f}{\partial z\partial y}-zy\frac{\partial^2f}{\partial x^2}+xy\frac{\partial^2f}{\partial z\partial x}\\
        \nonumber&-xz\frac{\partial^2f}{\partial y\partial x}+yz\frac{\partial^2f}{\partial x^2}+x^2\frac{\partial^2f}{\partial y\partial z}-y\frac{\partial f}{\partial z}-yx\frac{\partial^2f}{\partial x\partial z}\\
        =&z\frac{\partial f}{\partial y}-y\frac{\partial f}{\partial z}=\left(z\frac{\partial}{\partial y}-y\frac{\partial}{\partial z}\right)f(\vec{r})=-Q(a_1)f(\vec{r}),\\
        \nonumber[Q(a_3),Q(a_1)]=&\left(x\frac{\partial}{\partial y}-y\frac{\partial}{\partial x}\right)\left(y\frac{\partial f}{\partial z}-z\frac{\partial f}{\partial y}\right)-\left(y\frac{\partial}{\partial z}-z\frac{\partial}{\partial y}\right)\left(x\frac{\partial f}{\partial y}-y\frac{\partial f}{\partial x}\right)\\
        \nonumber=&x\frac{\partial f}{\partial z}+xy\frac{\partial^2f}{\partial y\partial z}-y^2\frac{\partial^2f}{\partial x\partial z}-xz\frac{\partial^2f}{\partial y^2}+yz\frac{\partial^2f}{\partial x\partial y}\\
        \nonumber&-yx\frac{\partial^2f}{\partial z\partial y}+zx\frac{\partial^2f}{\partial y^2}+y^2\frac{\partial^2f}{\partial z\partial x}-z\frac{\partial f}{\partial x}-zy\frac{\partial^2f}{\partial y\partial x}\\
        =&x\frac{\partial f}{\partial z}-z\frac{\partial f}{\partial x}=\left(x\frac{\partial}{\partial z}-z\frac{\partial}{\partial x}\right)f(\vec{r})=-Q(a_2)f(\vec{r}).
    \end{align}
    Due to the arbitrariness of $f(\vec{r})$, we have
    \begin{align}
        [Q(a_1),Q(a_2)]=&-Q(a_3),\\
        [Q(a_2),Q(a_3)]=&-Q(a_1),\\
        [Q(a_3),Q(a_1)]=&-Q(a_2).
    \end{align}
\end{sol}

\begin{prob}
    The generators of the real Lie algebra $L=\text{su}(2)$ are given by
    \[
        a_1=\frac{1}{2}\left(\begin{matrix}
            0&i\\
            i&0
        \end{matrix}\right),\quad a_2=\frac{1}{2}\left(\begin{matrix}
            0&1\\
            -1&0
        \end{matrix}\right),\quad a_3=\frac{1}{2}\left(\begin{matrix}
            i&0\\
            0&-i
        \end{matrix}\right).
    \]
    Show explicitly that $a_1$, $a_2$, and $a_3$ obey the commutation relations
    \begin{align*}
        [a_1,a_2]=&a_1a_2-a_2a_1=-a_3,\\
        [a_2,a_3]=&a_2a_3-a_3a_2=-a_1,\\
        [a_3,a_1]=&a_3a_1-a_1a_3=-a_2.
    \end{align*}
\end{prob}
\begin{sol}
    $a_1$, $a_2$, and $a_3$ obey the commutation relations:
    \begin{align}
        \nonumber[a_1,a_2]=&a_1a_2-a_2a_1=\frac{1}{4}\left(\begin{matrix}
            0&i\\
            i&0
        \end{matrix}\right)\left(\begin{matrix}
            0&1\\
            -1&0
        \end{matrix}\right)-\frac{1}{4}\left(\begin{matrix}
            0&1\\
            -1&0
        \end{matrix}\right)\left(\begin{matrix}
            0&i\\
            i&0
        \end{matrix}\right)=\frac{1}{4}\left(\begin{matrix}
            -i&0\\
            0&i
        \end{matrix}\right)-\frac{1}{4}\left(\begin{matrix}
            i&0\\
            0&-i
        \end{matrix}\right)\\
        =&\frac{1}{2}\left(\begin{matrix}
            -i&0\\
            0&i
        \end{matrix}\right)=-a_3,\\
        \nonumber[a_2,a_3]=&a_2a_3-a_3a_2=\frac{1}{4}\left(\begin{matrix}
            0&1\\
            -1&0
        \end{matrix}\right)\left(\begin{matrix}
            i&0\\
            0&-i
        \end{matrix}\right)-\frac{1}{4}\left(\begin{matrix}
            i&0\\
            0&-i
        \end{matrix}\right)\left(\begin{matrix}
            0&1\\
            -1&0
        \end{matrix}\right)=\frac{1}{4}\left(\begin{matrix}
            0&-i\\
            -i&0
        \end{matrix}\right)-\frac{1}{4}\left(\begin{matrix}
            0&i\\
            i&0
        \end{matrix}\right)\\
        =&\frac{1}{2}\left(\begin{matrix}
            0&-i\\
            -i&0
        \end{matrix}\right)=-a_1,\\
        \nonumber[a_3,a_1]=&a_3a_1-a_1a_3=\frac{1}{4}\left(\begin{matrix}
            i&0\\
            0&-i
        \end{matrix}\right)\left(\begin{matrix}
            0&i\\
            i&0
        \end{matrix}\right)-\frac{1}{4}\left(\begin{matrix}
            0&i\\
            i&0
        \end{matrix}\right)\left(\begin{matrix}
            i&0\\
            0&-i
        \end{matrix}\right)=\frac{1}{4}\left(\begin{matrix}
            0&-1\\
            1&0
        \end{matrix}\right)-\frac{1}{4}\left(\begin{matrix}
            0&1\\
            -1&0
        \end{matrix}\right)\\
        =&\frac{1}{2}\left(\begin{matrix}
            0&-1\\
            1&0
        \end{matrix}\right)=-a_2.
    \end{align}
\end{sol}

\begin{prob}
    The generators of the real Lie algebra $L=\text{su}(2)$ in the above problem can expressed in terms of the following Pauli matrices
    \[
        \sigma_1=\left(\begin{matrix}
            0&1\\
            1&0
        \end{matrix}\right),\quad\sigma_2=\left(\begin{matrix}
            0&-i\\
            i&0
        \end{matrix}\right),\quad\sigma_3=\left(\begin{matrix}
            1&0\\
            0&-1
        \end{matrix}\right).
    \]
    Show that the Pauli matrices possess the following properties.
    \begin{enumerate}
        \item[(a)] $\sigma_1^2=\sigma_2^2=\sigma_3^2$.
        \item[(b)] $\sigma_1\sigma_2=-\sigma_2\sigma_1=i\sigma_3$, $\sigma_2\sigma_3=-\sigma_3\sigma_2=i\sigma_1$, $\sigma_3\sigma_1=-\sigma_1\sigma_3=i\sigma_2$.
    \end{enumerate}
\end{prob}
\begin{sol}
    \begin{enumerate}
        \item[(a)] Since
        \begin{align}
            \sigma_1^2=&\left(\begin{matrix}
                0&1\\
                1&0
            \end{matrix}\right)\left(\begin{matrix}
                0&1\\
                1&0
            \end{matrix}\right)=\left(\begin{matrix}
                1&0\\
                0&1
            \end{matrix}\right),\\
            \sigma_2^2=&\left(\begin{matrix}
                0&-i\\
                i&0
            \end{matrix}\right)\left(\begin{matrix}
                0&-i\\
                i&0
            \end{matrix}\right)=\left(\begin{matrix}
                1&0\\
                0&1
            \end{matrix}\right)\\
            \sigma_3^2=&\left(\begin{matrix}
                1&0\\
                0&-1
            \end{matrix}\right)\left(\begin{matrix}
                1&0\\
                0&-1
            \end{matrix}\right)=\left(\begin{matrix}
                1&0\\
                0&1
            \end{matrix}\right),
        \end{align}
        we have
        \begin{equation}
            \sigma_1^2=\sigma_2^2=\sigma_3^2=\left(\begin{matrix}
                1&0\\
                0&1
            \end{matrix}\right).
        \end{equation}
        \item[(b)] Since
        \begin{align}
            \sigma_1\sigma_2=&\left(\begin{matrix}
                0&1\\
                1&0
            \end{matrix}\right)\left(\begin{matrix}
                0&-i\\
                i&0
            \end{matrix}\right)=\left(\begin{matrix}
                i&0\\
                0&-i
            \end{matrix}\right)=i\sigma_3,\\
            -\sigma_2\sigma_1=&-\left(\begin{matrix}
                0&-i\\
                i&0
            \end{matrix}\right)\left(\begin{matrix}
                0&1\\
                1&0
            \end{matrix}\right)=-\left(\begin{matrix}
                -i&0\\
                0&i
            \end{matrix}\right)=i\sigma_3,
        \end{align}
        we have
        \begin{equation}
            \sigma_1\sigma_2=-\sigma_2\sigma_1=i\sigma_3.
        \end{equation}
        Since
        \begin{align}
            \sigma_2\sigma_3=&\left(\begin{matrix}
                0&-i\\
                i&0
            \end{matrix}\right)\left(\begin{matrix}
                1&0\\
                0&-1
            \end{matrix}\right)=\left(\begin{matrix}
                0&i\\
                i&0
            \end{matrix}\right)=i\sigma_1,\\
            -\sigma_3\sigma_2=&-\left(\begin{matrix}
                1&0\\
                0&-1
            \end{matrix}\right)\left(\begin{matrix}
                0&-i\\
                i&0
            \end{matrix}\right)=-\left(\begin{matrix}
                0&-i\\
                -i&0
            \end{matrix}\right)=i\sigma_1,
        \end{align}
        we have
        \begin{equation}
            \sigma_2\sigma_3=-\sigma_3\sigma_2=i\sigma_1.
        \end{equation}
        Since
        \begin{align}
            \sigma_3\sigma_1=\left(\begin{matrix}
                1&0\\
                0&-1
            \end{matrix}\right)\left(\begin{matrix}
                0&1\\
                1&0
            \end{matrix}\right)=\left(\begin{matrix}
                0&1\\
                -1&0
            \end{matrix}\right)=i\sigma_2,\\
            -\sigma_1\sigma_3=-\left(\begin{matrix}
                0&1\\
                1&0
            \end{matrix}\right)\left(\begin{matrix}
                1&0\\
                0&-1
            \end{matrix}\right)=-\left(\begin{matrix}
                0&-1\\
                1&0
            \end{matrix}\right)=i\sigma_2,
        \end{align}
        we have
        \begin{equation}
            \sigma_3\sigma_1=-\sigma_1\sigma_3=i\sigma_2.
        \end{equation}
    \end{enumerate}
\end{sol}

\begin{prob}
    Let $\vec{n}=(n_1,n_2,n_3)$ be a unit vector specifying a direction in three dimensional space.
    \begin{enumerate}
        \item[(a)] Evaluate $(\vec{\sigma}\cdot\vec{n})^2$ with $\vec{\sigma}=(\sigma_1,\sigma_2,\sigma_3)$.
        \item[(b)] Evaluate $e^{i(\vec{\sigma}\cdot\vec{n})\omega/2}$.
    \end{enumerate}
\end{prob}
\begin{sol}
    \begin{enumerate}
        \item[(a)]
        \begin{align}
            \nonumber(\vec{\sigma}\cdot\vec{n})^2=&(n_1\sigma_1+n_2\sigma_2+n_3\sigma_3)^2\\
            \nonumber=&n_1^2\sigma_1^2+n_2^2\sigma_2^2+n_3^2\sigma_3^2+n_1n_2(\sigma_1\sigma_2+\sigma_2\sigma_1)+n_2n_3(\sigma_2\sigma_3+\sigma_3\sigma_2)+n_3n_1(\sigma_3\sigma_1+\sigma_1\sigma_3)\\
            \nonumber=&(n_1^2+n_2^2+n_3^2)\left(\begin{matrix}
                1&0\\
                0&1
            \end{matrix}\right)+n_1n_2\times 0+n_2n_3\times 0+n_3n_1\times 0\\
            =&\left(\begin{matrix}
                1&0\\
                0&1
            \end{matrix}\right).
        \end{align}
        \item[(b)] From
        \begin{equation}
            (\vec{\sigma}\cdot\vec{n})^2=\left(\begin{matrix}
                1&0\\
                0&1
            \end{matrix}\right),
        \end{equation}
        we know that
        \begin{equation}
            (\vec{\sigma}\cdot\vec{n})^j=\left\{\begin{array}{ll}
                \vec{\sigma}\cdot\vec{n}=\left(\begin{matrix}
                    n_3&n_1-in_2\\
                    n_1+in_2&-n_3
                \end{matrix}\right),&j\text{ is odd},\\
                \left(\begin{matrix}
                    1&0\\
                    0&1
                \end{matrix}\right),&j\text{ is even}.
            \end{array}\right.
        \end{equation}
        Then
        \begin{align}
            \nonumber e^{i(\vec{\sigma}\cdot\vec{n})\omega/2}=&1+\sum_{j=1}^{\infty}\frac{[i(\vec{\sigma}\cdot\vec{n})\omega/2]^j}{j!}\\
            \nonumber=&1+\sum_{j=0}^{\infty}\frac{i^{2j+1}(\vec{\sigma}\cdot\vec{n})^{2j+1}(\omega/2)^{2j+1}}{(2j+1)!}+\sum_{j=1}^{\infty}\frac{i^{2j}(\vec{\sigma}\cdot\vec{n})^{2j}(\omega/2)^{2j}}{(2j)!}\\
            \nonumber=&1+(\vec{\sigma}\cdot\vec{n})\sum_{j=0}^{\infty}\frac{i(-1)^j(\omega/2)^{2j+1}}{(2j+1)!}+\left(\begin{matrix}
                1&0\\
                0&1
            \end{matrix}\right)\sum_{j=1}^{\infty}\frac{(-1)^j(\omega/2)^{2j}}{(2j)!}\\
            \nonumber=&1+i\left(\begin{matrix}
                n_3&n_1-in_2\\
                n_1+in_2&-n_3
            \end{matrix}\right)\sin(\omega/2)+\left(\begin{matrix}
                1&0\\
                0&1
            \end{matrix}\right)[\cos(\omega/2)-1]\\
            =&\left(\begin{matrix}
                \cos(\omega/2)+in_3\sin(\omega/2)&(n_2+in_1)\sin(\omega/2)\\
                (-n_2+in_1)\sin(\omega/2)&\cos(\omega/2)-in_3\sin(\omega/2)
            \end{matrix}\right).
        \end{align}
    \end{enumerate}
\end{sol}
\end{document}
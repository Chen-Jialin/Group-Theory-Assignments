% !TEX program = pdflatex
% !TEX options = -synctex=1 -interaction=nonstopmode -file-line-error "%DOC%"
% Group Theory Assignment 04
\documentclass[UTF8,10pt,a4paper]{article}
\usepackage[scheme=plain]{ctex}
\newcommand{\CourseName}{Group Theory}
\newcommand{\CourseCode}{PHYS2102}
\newcommand{\Semester}{Spring, 2020}
\newcommand{\ProjectName}{Assignment 04}
\newcommand{\DueTimeType}{Due Time}
\newcommand{\DueTime}{8:15, March 31, 2020 (Wednesday)}
\newcommand{\StudentName}{陈稼霖}
\newcommand{\StudentID}{45875852}
\usepackage[vmargin=1in,hmargin=.5in]{geometry}
\usepackage{fancyhdr}
\usepackage{lastpage}
\usepackage{calc}
\pagestyle{fancy}
\fancyhf{}
\fancyhead[L]{\CourseName}
\fancyhead[C]{\ProjectName}
\fancyhead[R]{\StudentName}
\fancyfoot[R]{\thepage\ / \pageref{LastPage}}
\setlength\headheight{12pt}
\fancypagestyle{FirstPageStyle}{
    \fancyhf{}
    \fancyhead[L]{\CourseName\\
        \CourseCode\\
        \Semester}
    \fancyhead[C]{{\Huge\bfseries\ProjectName}\\
        \DueTimeType\ : \DueTime}
    \fancyhead[R]{Name : \makebox[\widthof{\StudentID}][s]{\StudentName}\\
        Student ID\@ : \StudentID\\
        Score : \underline{\makebox[\widthof{\StudentID}]{}}}
    \fancyfoot[R]{\thepage\ / \pageref{LastPage}}
    \setlength\headheight{36pt}
}
\usepackage{amsmath,amssymb,amsthm,bm}
\allowdisplaybreaks[4]
\newtheoremstyle{Problem}
{}
{}
{}
{}
{\bfseries}
{.}
{ }
{\thmname{#1}\thmnumber{ #2}\thmnote{ (#3)} Score: \underline{\qquad\qquad}}
\theoremstyle{Problem}
\newtheorem{prob}{Problem}
\newtheoremstyle{Solution}
{}
{}
{}
{}
{\bfseries}
{:}
{ }
{\thmname{#1}}
\makeatletter
\def\@endtheorem{\qed\endtrivlist\@endpefalse}
\makeatother
\theoremstyle{Solution}
\newtheorem*{sol}{Solution}
% \usepackage{graphicx}
\begin{document}
\thispagestyle{FirstPageStyle}
\begin{prob}
    The multiplication table for the group $D_3=\{E,D,F,A,B,C\}$ is given by
    \begin{table}[h]
        \centering
        \begin{tabular}{c|cccccc}
         & $E$ & $D$ & $F$ & $A$ & $B$ & $C$ \\ \hline
        $E$ & $E$ & $D$ & $F$ & $A$ & $B$ & $C$ \\
        $D$ & $D$ & $F$ & $E$ & $B$ & $C$ & $A$ \\
        $F$ & $F$ & $E$ & $D$ & $C$ & $A$ & $B$ \\
        $A$ & $A$ & $C$ & $B$ & $E$ & $F$ & $D$ \\
        $B$ & $B$ & $A$ & $C$ & $D$ & $E$ & $F$ \\
        $C$ & $C$ & $B$ & $A$ & $F$ & $D$ & $E$
        \end{tabular}
        \end{table}
        \begin{enumerate}
            \item[(a)] Determine the dimensions of all the inequivalent irreducible representations of $D_3$.
            \item[(b)] Find the character table for $D_3$.
        \end{enumerate}
\end{prob}
\begin{sol}
    \begin{enumerate}
        \item[(a)] The inverse of each element in $D_3$ are
        \begin{align}
            E^{-1}=&E,&D^{-1}=&F,&F^{-1}=&D,\\
            A^{-1}=&A,&B^{-1}=&B,&C^{-1}=&C.
        \end{align}
        Constructing a class from $D$: For $X=E,D,F$,
        \begin{equation}
            XDX^{-1}=D.
        \end{equation}
        For $X=A,B,C$,
        \begin{equation}
            XDX^{-1}=F.
        \end{equation}
        The class of $D_3$ constructed from $D$ is $\{D,F\}$.\\
        Using the similar method, we construct all the three classes of $D_3$:
        \[
            \mathcal{C}_1=\{E\},\quad\mathcal{C}_2=\{D,F\},\quad\mathcal{C}_3=\{A,B,C\}.
        \]
        The number of inequivalent irreducible representations of $D_3$ is equal to the number of classes of $G$, so $D_3$ has three inequivalent irreducible representations. Suppose the dimensions of the three inequivalent irreducible representations $\Gamma^1,\Gamma^2,\Gamma^3$ are $d_1,d_2,d_3$, respectively. The order of $D_3$ is $g=6$. The sum of the squares of the dimensions of the inequivalent irreducible representations of $D_3$ is equal to the order of $D_3$:
        \begin{equation}
            d_1^2+d_2^2+d_3^2=6.
        \end{equation}
        Solving the above equation, we get
        \begin{equation}
            d_1=1,\quad d_2=1,\quad d_3=2.
        \end{equation}
        \item[(b)] Since the character of identity in a representation is equal to the dimension of the representation, we have
        \begin{equation}
            \chi^1(\mathcal{C}_1)=\chi^2(\mathcal{C}_1)=1,\quad\chi^3(\mathcal{C}_1)=2.
        \end{equation}
        Since $A^2=B^2=C^2=E$, we have
        \begin{equation}
            \chi^p(\mathcal{C}_3)=\pm 1,\quad p=1,2.
        \end{equation}
        Since $AB=F$ and $AC=D$, we have
        \begin{equation}
            \chi^p(\mathcal{C}_2)=\chi^p(\mathcal{C}_3)^2=1.\quad p=1,2.
        \end{equation}
        There are two possibilities of $\chi^p(\mathcal{C}_3)$. Without loss of generality, we set $\Gamma^1$ to be the identity representation so that
        \begin{equation}
            \chi^1(\mathcal{C}_j)=1,\quad j=1,2,3.
        \end{equation}
        As for $\Gamma^2$, we have
        \begin{equation}
            \chi^2(\mathcal{C}_1)=1,\quad\chi^2(\mathcal{C}_2)=1,\quad\chi^3(\mathcal{C}_3)=-1.
        \end{equation}
        Using the orthogonality relation for characters
        \begin{equation}
            \frac{1}{g}\sum_{T\in G}\chi^q(T)^*\chi^q(T)=\delta_{pq},
        \end{equation}
        we have
        \begin{align}
            \label{1-1}\chi^3(\mathcal{C}_1)+2\chi^3(\mathcal{C}_2)+3\chi^3(\mathcal{C}_3)=&0\\
            \label{1-2}\chi^3(\mathcal{C}_1)+2\chi^3(\mathcal{C}_2)-3\chi^3(\mathcal{C}_3)=&0,\\
            \label{1-3}|\chi^3(\mathcal{C}_1)|^2+2|\chi^3(\mathcal{C}_2)|^2+3|\chi^3(\mathcal{C}_3)|^2=&6.
        \end{align}
        Adding up the two equations \eqref{1-1} and \eqref{1-2} above, we get
        \begin{equation}
            \chi^3(\mathcal{C}_1)+2\chi^3(\mathcal{C}_2)=0,
        \end{equation}
        so
        \begin{equation}
            \chi^3(\mathcal{C}_2)=-1.
        \end{equation}
        Using the equation \eqref{1-3}, we get
        \begin{equation}
            \chi^3(\mathcal{C}_3)=0.
        \end{equation}
        Now we have the character table for $D_3$ as shown in table \ref{1-CT}.
        \begin{table}[h]
            \centering
            \caption{character table for $D_3$}
            \label{1-CT}
            \begin{tabular}{c|ccc}
             & $\mathcal{C}_1=\{E\}$ & $\mathcal{C}_2=\{D,F\}$ & $\mathcal{C}_2=\{A,B,C\}$ \\ \hline
            $\Gamma^1$ & $1$ & $1$ & $1$ \\
            $\Gamma^2$ & $1$ & $1$ & $-1$ \\
            $\Gamma^3$ & $2$ & $-1$ & $0$
            \end{tabular}
            \end{table}
    \end{enumerate}
\end{sol}

\begin{prob}
    The transformation matrices in two-dimensional real space for the elements of the group $D_3=\{E,D,F,A,B,C\}$ are given by
    \begin{align*}
        R(E)=&\left(\begin{matrix}
            1&0\\
            0&1
        \end{matrix}\right),&R(D)=&\frac{1}{2}\left(\begin{matrix}
            -1&\sqrt{3}\\
            \sqrt{3}&-1
        \end{matrix}\right),&R(F)=&\frac{1}{2}\left(\begin{matrix}
            -1&\sqrt{3}\\
            -\sqrt{3}&-1
        \end{matrix}\right),\\
        R(A)=&\left(\begin{matrix}
            1&0\\
            0&1
        \end{matrix}\right),&R(B)=&\frac{1}{2}\left(\begin{matrix}
            -1&\sqrt{3}\\
            \sqrt{3}&-1
        \end{matrix}\right),&R(C)=&\frac{1}{2}\left(\begin{matrix}
            -1&\sqrt{3}\\
            -\sqrt{3}&1
        \end{matrix}\right).
    \end{align*}
    The basis function of a carrier space for $D_3$ are given by $\psi_1(\vec{\rho})=x^2F(\rho)$, $\psi_2(\vec{\rho})=xyF(\rho)$, and $\psi_3(\vec{\rho})=y^2F(\rho)$, where $F(\rho)$, a function of $\rho=\sqrt{x^2+y^2}$, ensures that the basis function as normalizable. Using
    \[
        Q(T)\psi_n(\vec{\rho})=\psi_n(R(T)^{-1}\vec{\rho})=\sum_{m=1}^3\Gamma(T)_{mn}\psi_m(\vec{\rho}),\quad n=1,2,3,
    \]
    find the representation matrix $\Gamma(F)$ of the element $F$ of $D_3$. Here $Q(T)$ is the scalar transformation operator and $\vec{\rho}$ is the position vector of a point in two-dimensional real space.
\end{prob}
\begin{sol}
    The inverse of $R(F)$ is equal to its transpose
    \begin{equation}
        R(F)^{-1}=R(F)^T=\frac{1}{2}\left(\begin{matrix}
            -1&-\sqrt{3}\\
            \sqrt{3}&-1
        \end{matrix}\right).
    \end{equation}
    Making $R(F)^{-1}$ operate on $\vec{\rho}$, we get
    \begin{equation}
        R(F)^{-1}\vec{\rho}=\frac{1}{2}\left(\begin{matrix}
            -1&-\sqrt{3}\\
            \sqrt{3}&-1
        \end{matrix}\right)\left(\begin{matrix}
            x\\
            y
        \end{matrix}\right)=\frac{1}{2}\left(\begin{matrix}
            -x-\sqrt{3}y\\
            \sqrt{3}x-y
        \end{matrix}\right)
    \end{equation}
    Making $Q(F)$ operating on $\psi_n(\vec{\rho})$, we get
    \begin{align}
        Q(F)\psi_1(\vec{\rho})=&\psi_1(R(F)^{-1}\vec{\rho})=\frac{1}{4}(x^2+2\sqrt{3}xy+3y^2)F(\rho)=\frac{1}{4}\psi_1(\vec{\rho})+\frac{\sqrt{3}}{2}\psi_2(\vec{\rho})+\frac{3}{4}\psi_3(\vec{\rho}),\\
        Q(F)\psi_2(\vec{\rho})=&\psi_2(R(F)^{-1}\vec{\rho})=\frac{1}{4}(-\sqrt{3}x^2-2xy+\sqrt{3}y^2)F(\rho)=-\frac{\sqrt{3}}{4}\psi_1(\vec{\rho})-\frac{1}{2}\psi_2(\vec{\rho})+\frac{\sqrt{3}}{4}\psi_3(\vec{\rho}),\\
        Q(F)\psi_3(\vec{\rho})=&\psi_3(R(F)^{-1}\vec{\rho})=\frac{1}{4}(3x^2-2\sqrt{3}xy+y^2)F(\rho)=\frac{3}{4}\psi_1(\vec{\rho})-\frac{\sqrt{3}}{2}\psi_2(\vec{\rho})+\frac{1}{4}\psi_3(\vec{\rho}).
    \end{align}
    Using
    \begin{equation}
        Q(T)\psi_n(\vec{\rho})=\psi_n(R(T)^{-1}\vec{\rho})=\sum_{m=1}^3\Gamma(T)_{mn}\psi_m(\vec{\rho}),\quad n=1,2,3,
    \end{equation}
    we find the representation matrix $\Gamma(F)$ of the element $F$ of $D_3$:
    \begin{equation}
        \Gamma(F)=\left(\begin{matrix}
            \frac{1}{4}&\frac{\sqrt{3}}{2}&\frac{3}{4}\\
            -\frac{\sqrt{3}}{4}&-\frac{1}{2}&\frac{\sqrt{3}}{4}\\
            \frac{3}{4}&-\frac{\sqrt{3}}{2}&\frac{1}{4}
        \end{matrix}\right).
    \end{equation}
\end{sol}

\begin{prob}
    Using the information given in the previous problem, find the representation matrix $\Gamma(B)$ of the element $B$ of $D_3$.
\end{prob}
\begin{sol}
    The inverse of $R(B)$ is its transpose
    \begin{equation}
        R(T)^{-1}=R(T)^T=\frac{1}{2}\left(\begin{matrix}
            -1&\sqrt{3}\\
            \sqrt{3}&1
        \end{matrix}\right).
    \end{equation}
    Making $R(B)^{-1}$ operate on $\vec{\rho}$, we get
    \begin{equation}
        R(F)^{-1}\vec{\rho}=\frac{1}{2}\left(\begin{matrix}
            -1&\sqrt{3}\\
            \sqrt{3}&1
        \end{matrix}\right)\left(\begin{matrix}
            x\\
            y
        \end{matrix}\right)=\frac{1}{2}\left(\begin{matrix}
            -x+\sqrt{3}y\\
            \sqrt{3}x+y
        \end{matrix}\right).
    \end{equation}
    Making $Q(B)$ operating on $\psi_n(\vec{\rho})$, we get
    \begin{align}
        Q(B)\psi_1(\vec{\rho})=&\psi_1(R(B)^{-1}\vec{\rho})=\frac{1}{4}(x^2-2\sqrt{3}xy+3y^2)F(\rho)=\frac{1}{4}\psi_1(\vec{\rho})-\frac{\sqrt{3}}{2}\psi_2(\vec{\rho})+\frac{3}{4}\psi_3(\vec{\rho}),\\
        Q(F)\psi_2(\vec{\rho})=&\psi_2(R(F)^{-1}\vec{\rho})=\frac{1}{4}(-\sqrt{3}x^2+2xy+\sqrt{3}y^2)F(\rho)=-\frac{\sqrt{3}}{4}\psi_1(\vec{\rho})+\frac{1}{2}\psi_2(\vec{\rho})+\frac{\sqrt{3}}{4}\psi_3(\vec{\rho}),\\
        Q(F)\psi_3(\vec{\rho})=&\psi_3(R(F)^{-1}\vec{\rho})=\frac{1}{4}(3x^2+2\sqrt{3}xy+y^2)F(\rho)=\frac{3}{4}\psi_1(\vec{\rho})+\frac{\sqrt{3}}{2}\psi_2(\vec{\rho})+\frac{1}{4}\psi_3(\vec{\rho}).
    \end{align}
    Using
    \begin{equation}
        Q(T)\psi_n(\vec{\rho})=\psi_n(R(T)^{-1}\vec{\rho})=\sum_{m=1}^3\Gamma(T)_{mn}\psi_m(\vec{\rho}),\quad n=1,2,3,
    \end{equation}
    we find the representation matrix $\Gamma(F)$ of the element $F$ of $D_3$:
    \begin{equation}
        \Gamma(F)=\left(\begin{matrix}
            \frac{1}{4}&-\frac{\sqrt{3}}{2}&\frac{3}{4}\\
            -\frac{\sqrt{3}}{4}&\frac{1}{2}&\frac{\sqrt{3}}{4}\\
            \frac{3}{4}&\frac{\sqrt{3}}{2}&\frac{1}{4}
        \end{matrix}\right).
    \end{equation}
\end{sol}

\begin{prob}
    Show that, if the projection operators $P_{mn}^p$ and $P_{jk}^q$ belong to two unitary irreducible representations $\Gamma^p$ and $\Gamma^q$ of $G$ that are not equivalent if $p\neq q$ (but are identical if $p=q$), then $P_{mn}^pP_{jk}^q=\delta_{pq}\delta_{nj}P_{mk}^q$.
\end{prob}
\begin{sol}
    Using the definition of the projection operators
    \begin{align}
        P_{mn}^p=&\frac{d_p}{g}\sum_{T\in G}\Gamma^p(T)_{mn}^*Q(T),\\
        P_{jk}^q=&\frac{d_q}{g}\sum_{T'\in G}\Gamma^p(T')_{jk}^*Q(T'),
    \end{align}
    we have
    \begin{equation}
        P_{mn}^pP_{jk}^q=\frac{d_pd_q}{g^2}\sum_{T,T'\in G}\Gamma^p(T)_{mn}^*\Gamma^q(T')_{jk}^*Q(T)Q(T').
    \end{equation}
    Since
    \begin{equation}
        Q(T)Q(T')=Q(TT'),
    \end{equation}
    we have
    \begin{equation}
        P_{mn}^pP_{jk}^q=\frac{d_pd_q}{g^2}\sum_{T,T'\in G}\Gamma^p(T)_{mn}^*\Gamma^q(T')_{jk}^*Q(TT').
    \end{equation}
    Setting $T''=TT'\in G$ and replacing $T'$ with $T'=T^{-1}T''$, we get
    \begin{equation}
        P_{mn}^pP_{jk}^q=\frac{d_pd_q}{g^2}\sum_{T,T''\in G}\Gamma^p(T)_{mn}^*\Gamma^q(T^{-1}T'')_{jk}^*Q(T'').
    \end{equation}
    Since $\Gamma^p$ and $\Gamma^q$ are two unitary representations,
    \begin{equation}
        \Gamma^q(T^{-1}T'')^*=[\Gamma^q(T^{-1})\Gamma^q(T'')]^*=[\Gamma^q(T)^{-1}\Gamma^q(T'')]^*=[\Gamma^q(T)^{\dagger}\Gamma^q(T'')]^*=\Gamma^q(T)^T\Gamma^q(T'')^*,
    \end{equation}
    we have
    \begin{equation}
        P_{mn}^pP_{jk}^q=\frac{d_pd_q}{g^2}\sum_{T,T''\in G}\sum_l\Gamma^p(T)_{mn}^*\Gamma^q(T)_{lj}\Gamma^q(T'')_{lk}Q(T'').
    \end{equation}
    Using the orthogonality relation for unitary irreducible representations
    \begin{equation}
        \frac{1}{g}\sum_{T\in G}\Gamma^p(T)_{mn}^*\Gamma^q(T)_{lj}=\frac{1}{d_p}\delta_{pq}\delta_{ml}\delta_{nj},
    \end{equation}
    we get
    \begin{equation}
        P_{mn}^pP_{jk}^q=\frac{d_q}{g}\delta_{pq}\delta_{nj}\sum_{T''\in G}\sum_l\delta_{ml}\Gamma^q(T'')_{lk}Q(T'')=\frac{d_q}{g}\delta_{pq}\delta_{nj}\sum_{T''\in G}\Gamma^q(T'')_{mk}Q(T'')=\delta_{pq}\delta_{nj}P_{mk}^q.
    \end{equation}
\end{sol}

\begin{prob}
    Choosing $\phi(\vec{r})=(xy+yz)e^{-r}$, construct the basis function for two-dimensional irreducible representation $\Gamma^5$ of the crystallographic point group $D_4$.
\end{prob}
\begin{sol}
    The transformation matrices of $D_4$ are
    \begin{align}
        R(E)=&\left(\begin{matrix}
            1&0&0\\
            0&1&0\\
            0&0&1
        \end{matrix}\right),&R(C_{2x})=&\left(\begin{matrix}
            1&0&0\\
            0&-1&0\\
            0&0&-1
        \end{matrix}\right),&R(C_{2y})=&\left(\begin{matrix}
            -1&0&0\\
            0&1&0\\
            0&0&-1
        \end{matrix}\right),&R(C_{2z})=&\left(\begin{matrix}
            -1&0&0\\
            0&-1&0\\
            0&0&1
        \end{matrix}\right),\\
        R(C_{4z})=&\left(\begin{matrix}
            0&0&-1\\
            0&1&0\\
            1&0&0
        \end{matrix}\right),&R(C_{4y}^{-1})=&\left(\begin{matrix}
            0&0&1\\
            0&1&0\\
            -1&0&0
        \end{matrix}\right),&R(C_{2c})=&\left(\begin{matrix}
            0&0&1\\
            0&-1&0\\
            1&0&0
        \end{matrix}\right),&R(C_{2d})=&\left(\begin{matrix}
            0&0&-1\\
            0&-1&0\\
            -1&0&0
        \end{matrix}\right).
    \end{align}
    The inverses of these transformation operators are their transposes respectively
    \begin{align}
        R(E)^{-1}=&R(E)^T=\left(\begin{matrix}
            1&0&0\\
            0&1&0\\
            0&0&1
        \end{matrix}\right),&R(C_{2x})^{-1}=&R(C_{2x})=\left(\begin{matrix}
            1&0&0\\
            0&-1&0\\
            0&0&-1
        \end{matrix}\right),\\
        R(C_{2y})^{-1}=&R(C_{2y})^T=\left(\begin{matrix}
            -1&0&0\\
            0&1&0\\
            0&0&-1
        \end{matrix}\right),&R(C_{2z})^{-1}=&R(C_{2z})^T=\left(\begin{matrix}
            -1&0&0\\
            0&-1&0\\
            0&0&1
        \end{matrix}\right),\\
        R(C_{4y})^{-1}=&R(C_{4y})^T=\left(\begin{matrix}
            0&0&1\\
            0&1&0\\
            -1&0&0
        \end{matrix}\right),&R(C_{4y}^{-1})^{-1}=&R(C_{4y}^{-1})^T=\left(\begin{matrix}
            0&0&-1\\
            0&1&0\\
            1&0&0
        \end{matrix}\right),\\
        R(C_{2c})^{-1}=&R(C_{2c})^T=\left(\begin{matrix}
            0&0&1\\
            0&-1&0\\
            1&0&0
        \end{matrix}\right),&R(C_{2d})^{-1}=&R(C_{2d})^T=\left(\begin{matrix}
            0&0&-1\\
            0&-1&0\\
            -1&0&0
        \end{matrix}\right).
    \end{align}
    Making the inverse of transformation operators operate on $\vec{r}$, we get
    \begin{align}
        R(E)^{-1}\vec{r}=&\left(\begin{matrix}
            1&0&0\\
            0&1&0\\
            0&0&1
        \end{matrix}\right)\left(\begin{matrix}
            x\\
            y\\
            z
        \end{matrix}\right)=\left(\begin{matrix}
            x\\
            y\\
            z
        \end{matrix}\right),\\
        R(C_{2x})^{-1}\vec{r}=&\left(\begin{matrix}
            1&0&0\\
            0&-1&0\\
            0&0&-1
        \end{matrix}\right)\left(\begin{matrix}
            x\\
            y\\
            z
        \end{matrix}\right)=\left(\begin{matrix}
            x\\
            -y\\
            -z
        \end{matrix}\right),\\
        R(C_{2y})^{-1}\vec{r}=&\left(\begin{matrix}
            -1&0&0\\
            0&1&0\\
            0&0&-1
        \end{matrix}\right)\left(\begin{matrix}
            x\\
            y\\
            z
        \end{matrix}\right)=\left(\begin{matrix}
            -x\\
            y\\
            -z
        \end{matrix}\right),\\
        R(C_{2z})^{-1}\vec{r}=&\left(\begin{matrix}
            -1&0&0\\
            0&-1&0\\
            0&0&1
        \end{matrix}\right)\left(\begin{matrix}
            x\\
            y\\
            z
        \end{matrix}\right)=\left(\begin{matrix}
            -x\\
            -y\\
            z
        \end{matrix}\right),\\
        R(C_{4y})^{-1}\vec{r}=&\left(\begin{matrix}
            0&0&1\\
            0&1&0\\
            -1&0&0
        \end{matrix}\right)\left(\begin{matrix}
            x\\
            y\\
            z
        \end{matrix}\right)=\left(\begin{matrix}
            z\\
            y\\
            -x
        \end{matrix}\right),\\
        R(C_{4y}^{-1})^{-1}\vec{r}=&\left(\begin{matrix}
            0&0&-1\\
            0&1&0\\
            1&0&0
        \end{matrix}\right)\left(\begin{matrix}
            x\\
            y\\
            z
        \end{matrix}\right)=\left(\begin{matrix}
            -z\\
            y\\
            x
        \end{matrix}\right),\\
        R(C_{2c})^{-1}\vec{r}=&\left(\begin{matrix}
            0&0&1\\
            0&-1&0\\
            1&0&0
        \end{matrix}\right)\left(\begin{matrix}
            x\\
            y\\
            z
        \end{matrix}\right)=\left(\begin{matrix}
            z\\
            -y\\
            x
        \end{matrix}\right),\\
        R(C_{2d})^{-1}\vec{r}=&\left(\begin{matrix}
            0&0&-1\\
            0&-1&0\\
            -1&0&0
        \end{matrix}\right)\left(\begin{matrix}
            x\\
            y\\
            z
        \end{matrix}\right)=\left(\begin{matrix}
            -z\\
            -y\\
            -x
        \end{matrix}\right).
    \end{align}
    Making the scalar transformation operators operate on $\phi(\vec{r})$, we get
    \begin{align}
        Q(E)\phi(\vec{r})=&\phi(R(E)^{-1}\vec{r})=(xy+yz)e^{-r},\\
        Q(R_{2x})\phi(\vec{r})=&\phi(R(C_{2x})^{-1}\vec{r})=(-xy+yz)e^{-r},\\
        Q(R_{2y})\phi(\vec{r})=&\phi(R(C_{2y})^{-1}\vec{r})=(-xy-yz)e^{-r},\\
        Q(R_{2z})\phi(\vec{r})=&\phi(R(C_{2z})^{-1}\vec{r})=(xy-yz)e^{-r},\\
        Q(R_{4y})\phi(\vec{r})=&\phi(R(C_{4y})^{-1}\vec{r})=(-xy+yz)e^{-r},\\
        Q(R_{4y}^{-1})\phi(\vec{r})=&\phi(R(C_{4y}^{-1})^{-1}\vec{r})=(xy-yz)e^{-r},\\
        Q(R_{2c})\phi(\vec{r})=&\phi(R(C_{2c})^{-1}\vec{r})=(-xy-yz)e^{-r},\\
        Q(R_{2d})\phi(\vec{r})=&\phi(R(C_{2d})^{-1}\vec{r})=(xy+yz)e^{-r}.
    \end{align}
    The two-dimensional irreducible representation $\Gamma^5$ of $D_4$ is
    \begin{align}
        \Gamma^5(E)=&\left(\begin{matrix}
            1&0\\
            0&1
        \end{matrix}\right),&\Gamma^5(C_{2x})=&\left(\begin{matrix}
            1&0\\
            0&-1
        \end{matrix}\right),&\Gamma^5(C_{2y})=&\left(\begin{matrix}
            -1&0\\
            0&-1
        \end{matrix}\right),&\Gamma^5(C_{2z})=&\left(\begin{matrix}
            -1&0\\
            0&1
        \end{matrix}\right),\\
        \Gamma^5(C_{4y})=&\left(\begin{matrix}
            0&-1\\
            1&0
        \end{matrix}\right),&\Gamma^5(C_{4y}^{-1})=&\left(\begin{matrix}
            0&1\\
            -1&0
        \end{matrix}\right),&\Gamma^5(C_{2c})=&\left(\begin{matrix}
            0&1\\
            1&0
        \end{matrix}\right),&\Gamma^5(C_{2d})=&\left(\begin{matrix}
            0&-1\\
            -1&0
        \end{matrix}\right).
    \end{align}
    Using
    \begin{equation}
        P_{mn}^p(\vec{r})=\frac{d_p}{g}\sum_{T\in D_4}\Gamma^p(T)_{mn}Q(T)\phi(\vec{r}),
    \end{equation}
    we have
    \begin{align}
        P_{11}^5\phi(\vec{r})=&\frac{1}{4}[(xy+yz)+(-xy+yz)-(-xy-yz)-(xy-yz)]e^{-r}=yze^{-r},\\
        P_{22}^5\phi(\vec{r})=&\frac{1}{4}[(xy+yz)-(-xy+yz)-(-xy-yz)+(xy-yz)]e^{-r}=xye^{-r}.
    \end{align}
    We calculate the coefficients
    \begin{align}
        \nonumber(c_1^5)^2=&(P_{11}^5\phi(\vec{r}),P_{11}^5\phi(\vec{r}))^{1/2}=\iiint_{-\infty}^{+\infty}dx\,dy\,dz\,y^2z^2e^{-2r}\\
        =&\int_0^{2\pi}d\varphi\int_0^{\pi}\sin\theta\,d\theta\int_0^{+\infty}r^2\,dr\,(r\sin\theta\sin\varphi)^2(r\cos\theta)^2e^{-2r}=\frac{3}{2}\pi,\\
        c_2^5=&(P_{11}^5(\vec{r}),P_{22}^5\phi(\vec{r}))=\iiint_{-\infty}^{+\infty}dx\,dy\,dz\,x^2y^2e^{-2r}=(c_1^5)^2=\frac{3}{2}\pi.
    \end{align}
    Without loss of generality, we set $c_1^5=\left(\frac{3}{2}\pi\right)^{1/2}$.
    The basis functions for $\Gamma^5$ of $D_4$ are
    \begin{align}
        \psi_1^5(\vec{r})=&\frac{P_{11}^p\phi(\vec{r})}{c_1^5}=\left(\frac{2}{3\pi}\right)^{1/2}yze^{-r},\\
        \psi_2^5(\vec{r})=&P_{21}^5\psi_1^5(\vec{r})=\frac{2}{8}\sum_{T\in G}\Gamma^5(T)_{21}^*Q(T)\psi(\vec{r})=\frac{1}{4}\left(\frac{2}{3\pi}\right)^{1/2}[(-xy)-xy+(-xy)-xy]e^{-1}=-\left(\frac{2}{3\pi}\right)^{1/2}xye^{-r}.
    \end{align}
\end{sol}
\end{document}